\begin{introduction}

这是一份非正式的``北京大学论文文档类''的说明.

\section*{问题的背景}

我逛未名的时间并不长. Thesis~和~MathTools~算是两个常去的版.
~dypang\cite{dypang}~和~FerretL\cite{FerretL}~所作的论文模版
在置顶那里挂了一定时间的时候, 我自己毕业的年份也就到了.
MathTools~版的~lwolf\cite{lwolf}~和~Langpku\cite{Langpku}等
显然对~dypang~的工作做了很多改进, 但是这个文档类仍有一些不足.

另外, 学校的文件又有变化, 比如\cite{F11-2007}就与以前不同了.

总之, 到了需要改变某些东西的时候.

\section*{当前的进展}

写作北京大学论文文档类的工作, 就我所知, 从~dypang~开始.
该版本改自清华模版. FerretL~则以另一个清华模版为基础(其中还可
看到西安交大的痕迹:-).

这两个版本的共同缺点是加入了太多不必要的代码. 后续的改进并没有
改变这一点.

\section*{思路和结果}

因此, 我希望写作一个新的文档类, 它只做一些大家都用得着的设置.
因为与清华等学校不同, 北大于论文正文的格式并没有作详细的规定.
\verb|pkuthss.cls|~就是目前的成品. 我冒昧地称它为北京大学论文
文档类.

\section*{行文结构}

本文将首先从源代码出发介绍这个文档类的基本情况, 然后用实例说明
这个文档类的使用方法, 最后指出目前存在的问题和努力的方向.

\end{introduction}
