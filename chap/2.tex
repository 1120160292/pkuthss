\chapter{用法示例}

下面将一步步演示~pkuthss.cls~的用法.

\section{开始写作}

自然地, 我们要导入这个文档类
\begin{verbatim}
\documentclass{pkuthss}
\end{verbatim}

在开始正文之前, 我们需要自定义一些内容.
比如, 如果我需要打数学公式或定理等, 就需要对它们
作一些设置.

\section{数学部分}

下面是我的常用装备:
\begin{enumerate}
\item \verb|amsmath|~系列, 是一般数学公式必备的;
\item \verb|ntheorem|, 用于自定义定理环境;
\item \verb|xy-pic|, 用于作交换图表.
\end{enumerate}

具体地, 下述代码就是一例:
\begin{verbatim}
\usepackage{amsmath,amssymb,amsfonts}
\usepackage[amsmath,thmmarks,hyperref]{ntheorem}
\usepackage[all,import]{xy}

\renewcommand{\theequation}{\thechapter.\arabic{equation}}
\numberwithin{equation}{chapter}   %% 设置方程分章编号

% 从1.0rc2版以后, preamble可以直接使用中文.
\theorembodyfont{\kaishu}
\theoremheaderfont{\heiti}
\newtheorem{definition}{定义}[chapter]
\newtheorem{lemma}[definition]{引理}
\newtheorem{proposition}[definition]{命题}
\newtheorem{theorem}[definition]{定理}
\newtheorem{corollary}[definition]{推论}
\theorembodyfont{\songti}
\newtheorem{example}[definition]{例}
\newtheorem*{remark}{注\ }
\theoremsymbol{$\square$}
\newtheorem*{proof}{证明}
\end{verbatim}

为方便使用, 以上代码被放在文件~\verb|chaps/mydefs.tex|~中.

\section{其他部分}

对于非数学专业, 我觉得使用~\LaTeX~应该也是很正常的, 这也是我
不在文档类中加入上述数学相关代码的原因.

作为一个小的示例, 我们来看~\verb|chaps/mydefs.tex|~中的一个设置:
\begin{center}
\begin{verbatim}
\usepackage[cross,a4,center]{crop}
\end{verbatim}
\end{center}
\verb|crop|~宏包是用来产生裁剪线的(请观察~\verb|sample.pdf|~的页面四个角),
如果不希望产生这些线条, 可注释掉这一行. 这可以方便地用于区分修改版和定稿版.

另外, 在~\verb|chaps/mydefs.tex|~中, \verb|hyperref|~宏包用来产生链接和书签.
如果不需要这些, 也可注释掉这部分.

\LaTeX~的宏包非常丰富, 推荐阅读~\cite{ch8}.

做完这些设置以后, 就可以开始写作正文了. 具体可参见~\verb|sample.tex|.
