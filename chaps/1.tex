\chapter{源代码说明}

本章对~pkuthss.cls~和~pkulogo.ps~两个文件的内容作一定的说
明.

\section{pkuthss.cls}

沿袭~PKUthesis~的思路, pkuthss.cls~直接调用~ctex~宏包,
它的目的是产生一个恰当的写作环境. 除了代替作者完成论文开头和
结尾的一些~dirty code~之外, 在这个文件中本质上只做了三处设置.
即设置页面布局、标题样式和页眉页脚.

\subsection{页面布局}
我们用~\verb|geometry|~宏包来完成这个工作.
\begin{verbatim}
\geometry{a4paper,centering,height=240mm,width=150mm,%
          includeheadfoot,headheight=12.65pt}
\end{verbatim}
它设置版心尺寸为~240mm x 150mm, 选项~\verb|includeheadfoot|~表明
这个版心尺寸包含页眉页脚.

至于另两个选项, \verb|centering|~表明文字内容处于纸张中心,
指定这个~\verb|headheight|~是为了消除编译时产生的一个~Warning.
如果需要修改这些参数或添加其它选项, 请参考~\cite{geometry}.

\subsection{标题样式}
直接使用~ctex~提供的~\verb|\CTEXsetup|~命令. 没什么好说的.
\begin{verbatim}
\CTEXsetup[format={\bf\zihao{-3}\centering},
           beforeskip={5mm},
           afterskip={1ex plus .2ex}]{chapter}
\CTEXsetup[format={\bf\zihao{4}\centering},
           beforeskip={-3ex plus -1ex minus -.2ex},
           afterskip={1ex plus .2ex}]{section}
\CTEXsetup[format={\bf\zihao{-4}},
           indent={2\ccwd},
           beforeskip={-2.5ex plus -1ex minus -.2ex},
           afterskip={1ex plus .2ex}]{subsection}
\end{verbatim}
它设置章的标题和节的标题都居中, 小节的标题左起缩进.
要了解其它选项, 请参考~\cite{ctex}~和~\cite{ctexfaq}.

\subsection{页眉页脚}
使用~\verb|fancyhdr|~宏包,
\begin{verbatim}
\fancypagestyle{plain}{%
  \fancyhf{}%
  \renewcommand{\headrulewidth}{0pt}%
  \renewcommand{\footrulewidth}{0pt}}
\pagestyle{fancy}
  \fancyhf{}
  \fancyhead[LE]{\small\songti\rightmark}
  \fancyhead[RO]{\small\songti\leftmark}
  \fancyhead[LO,RE]{\small\songti\@university\label@thesis}
  \fancyfoot[RO,LE]{\small --~\thepage~--}
\end{verbatim}
其含义也是不言自明的.

\subsection{其他}
剩下的~dirty code~似乎没什么好说. 它们主要参考了~dypang\cite{dypang}~和
~\cite{texbook}.

\section{pkulogo.ps}

之所以要说明这个文件, 是因为它并不是官方的版本. 但它是我写出
来的最好的一个版本.

在~pkulogo.ps~文件中, 跳过以~\%~开始的一些注释行,
我们看到
\begin{verbatim}
    0.0 1.0 1.0 0.5 setcmykcolor
\end{verbatim}

这个语句设置北大红的色值\footnotemark.
\footnotetext{感谢~MathTools~版~mycc~提供信息.}

在将坐标原点移到图片的中心之后, 以下依次用~arc~命令作出两个圆:
\begin{verbatim}
newpath
6 setlinewidth
0 0 195 0 360 arc
stroke
newpath
4 setlinewidth
0 0 143 0 360 arc
stroke
\end{verbatim}

接下来的一大段数据(22--75~行)用于描出``北大''字样. 第~77~行则
作出包围此图案的圆形. 这一段的关键在于第~78~行的~\verb|eofill|,
它表示对刚作出的线条执行奇偶相间的填充. 这样, 中间空出的``北大''
字样就是透明的. 从而即使页面带背景色, 插入这个图片时也会很正
常(例如在答辩幻灯中使用).

最后一段用于添加文字~PEKING UNIVERSITY 1898, 应该说是直接了当的.

顺便说下, 直接使用~pkulogo.ps~作为论文封面图片也是可行的. 因为
它使用的~Times-Bold~字体一般操作系统都有, 这样用
~\LaTeX\ + dvipdfm~方式是可以编译成功的,
但~\LaTeX\ + dvips + ps2pdf~方式则不行(关于这两种编译方式,
详见~\S\ref{sec:compile}).

为了兼容后面这种编译方式, 可以使用~\verb|ghostscript|~将~
\verb|ps|~文件转换为~\verb|eps|~和~\verb|pdf|~格式.
参见~\verb|imgs|~目录下的~\verb|make.bat|~和~\verb|Makefile|~.

至于另外的文件, pkuword.ps~是修改~dypang~版本得到的, 用了
与~pkulogo.ps~同样的手法, 使空处为透明. 这样, 要想把它改为黑色,
只需将第~9~行改为
\begin{verbatim}
    0 setgray
\end{verbatim}
它与~dypang~版本的另一个区别是, 用于描画四个字的代码分
别被集合到了一起, 这样方便以后作其他用途时分别处理.
