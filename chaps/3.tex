\chapter{使用须知}\label{chap:notes}

本目录下各文件的内容见表\ref{tab:filelist}.
\begin{table}[h]
\begin{tabular}{ll}\hline
   pkuthss.cls  & --- 北京大学博士学位论文文档类;\\
   pkuthss.def  & --- 定义文件, 在~pkuthss.cls~中用到;\\
   sample.tex   & --- 主文件, 编译该文件即可;\\
   sample.pdf   & --- 即您正在看的这个文件, 是编译~sample.tex~所得;\\
   Makefile     & --- Makefile for linux/unix;\\
   make.bat     & --- `Makefile' for MS Windows;\\
   imgs/        & --- 文件夹, 包含论文中所有图片和封面logo;\\
   chaps/       & --- 文件夹, 包含各章节内容和参考文献.\\ \hline
\end{tabular}
\caption{文件列表}\label{tab:filelist}
\end{table}

\section{软件需求}\label{sec:requirement}

正确编译需要
\begin{enumerate}
\item 一个基本的~\TeX~发行版.
\item CJK~或~CCT~宏包. 其安装和配置散见于各大论坛, 容易搜索得到.
\item ctex~宏包. 下载和安装见~http://www.ctex.org/PackageCTeX.
\item sample.tex~的默认编译方式需要~dvipdfmx.
\end{enumerate}

当然, CTeX套装Basic版就已经包含所有这些了.
TeXLive~也包含~CJK. 安装~ctex~宏包相对容易.

\section{编译}\label{sec:compile}

本文档类支持三种编译方式, 即
\begin{itemize}
  \item \LaTeX\ + dvips + ps2pdf~方式:
    即顺次执行 latex, bibtex, latex, latex, dvips, ps2pdf. 在使用此种
    方式时, 有些宏包的~driver~选项需要设置为~dvips. 例如~sample.tex~使用
    这种方式编译时, 需要将~\verb|chaps/mydefs.tex|~第~9~行的~dvipdfm~改为~dvips.

  \item \LaTeX\ + dvipdf~方式:
    即顺次执行 latex, bibtex, latex, latex, dvipdfmx. 类似于上一种方式,
    相应的~driver~选项一般为~dvipdf~或~dvipdfm.

  \item pdf\LaTeX~方式:
    即顺次执行 pdflatex, bibtex, gbk2uni(可选), pdflatex, pdflatex. 类似于
    上一种方式, 相应的~driver~选项为~pdftex.
\end{itemize}
Windows~的默认编译方式~(即使用~make.bat~)是第二种.
Linux~的默认编译方式~(即使用~Makefile)~是第三种, 在这方式下如果要用
~\verb|hyperref|~产生书签, 需要执行一下~\verb|gbk2uni|.
\section{FAQ}

{\bf Q:} 我不喜欢``绪言'', 而想改作``序言''(或``绪论''或其他), 怎么办?

{\bf A:} \verb|\renewcommand{\introductionname}{序言}|

{\bf Q:} 默认的是博士研究生论文, 我想用它来做本科论文, 可以吗?

{\bf A:} \verb|\renewcommand{\thesisname}{学士毕业论文}|

{\bf Q:} 我的编译结果很奇怪, 文字很靠近页面的顶端. 是怎么回事?

{\bf A:} 请检查你的程序设置. 如果使用~WinEdt, 可点击~Options,
选择~Execution Modes, 检查一下~dvi2ps, dvi2pdf, ps2pdf~等程序的纸张设置.


\section{问题和展望}

应该注意到, 研究生手册中的\cite{F13}和其电子版
\begin{center}
http://grs.pku.edu.cn/py/content/F\_13.doc
\end{center}
要求的论文封面并不一致. 这里以网络版为准.

由于时间无多, 将不考虑对本文档作进一步的更新. 但对于使用中遇到的问题, 在
今年~6~月之前, 我将尽量解决.

\section{更新记录}

\verbatiminput{ChangeLog.txt}
