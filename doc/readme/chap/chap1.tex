% vim:ts=4:sw=4
%
% Documentation for pkuthss.
%
% Copyright (c) 2008-2009 solvethis
% Copyright (c) 2010-2015 Casper Ti. Vector
%
% This work may be distributed and/or modified under the conditions of the
% LaTeX Project Public License, either version 1.3 of this license or (at
% your option) any later version.
% The latest version of this license is in
%   http://www.latex-project.org/lppl.txt
% and version 1.3 or later is part of all distributions of LaTeX version
% 2005/12/01 or later.
%
% This work has the LPPL maintenance status `maintained'.
% The current maintainer of this work is Casper Ti. Vector.
%
% This work consists of the following files:
%   pkuthss.tex
%   chap/copyright.tex
%   chap/abstract.tex
%   chap/introduction.tex
%   chap/chap1.tex
%   chap/chap2.tex
%   chap/chap3.tex
%   chap/conclusion.tex
%   chap/encl1.tex
%   chap/acknowledge.tex

\chapter{使用介绍}
	\section{系统要求}\label{sec:req}

	正确编译需要以下几部分:
	\begin{itemize}
		\item 一个基本的 \hologo{TeX} 发行版。
		\item CJK 或 XeCJK(供 \hologo{XeLaTeX} 使用)宏包。
		\item ctex 宏包\supercite{ctex,ctex-faq}(提供了 ctexbook 文档类)。
		\item 中文字体。
		\item 如果要使用 biblatex 进行文献列表和引用的排版的话,
			还需要 biblatex 宏包\supercite{biblatex};
		\item 如果需要对中文文献进行按汉语拼音的排序的话,
			还需要 biber 程序\supercite{biber}。
		\item 如果使用默认的文献列表和引用样式的话,
			还需要作者编写的 biblatex 样式
			(biblatex-caspervector)\supercite{biblatex-caspervector}。
		\item 如果需要使用 Makefile 来实现自动编译,还需要 Make 工具;
			但如果使用由批处理实现的伪“Makefile”就不用了。
	\end{itemize}

	\myemph{最新}的\myemph{完全版} \hologo{TeX} Live 系统(\myemph{%
		注:
		某些 Linux 发行版软件仓库中的 \hologo{TeX} Live 有问题,
		建议使用独立安装版的 \hologo{TeX} Live%
		\unemph{\footnote{\url{http://www.tug.org/texlive/}.}}。%
	})都已经包含除中文字体和 Make 之外所有要求的项目。%
	\myemph{%
		为了获得最好的支持,
		我们建议用户使用最新完全版的 \hologo{TeX} 系统和各宏包。%
	}

	中文字体需要用户自行获得。\myemph{%
		注:
		一些中文字体的字库不全,
		只有 GB2312 字符集内字符的字体信息。
		这种情况通常会造成编译生成的 pdf 文件中缺少部分字符,
		其中一种典型症状是封面的“〇”字显示不出来。
		如果要使用中易公司的字体,
		则建议使用 Windows Vista 及其以后版本提供的%
		宋体、黑体、楷体和仿宋体,
		以及 Microsoft Office 2003 及其以后版本提供的隶书和幼圆体,
		这些字体是 GB18030 字符集的,不存在上述问题。%
	}

	Linux 用户可以从软件源获得 GNU 的 make;
	其它类 UNIX 系统应该也会提供 make 工具,请参阅相应的文档以获得帮助。%
	Windows 用户可以从以下地址下载 Windows 下的 GNU make 工具:\\
	\hspace*{\parindent}%
	\url{http://gnuwin32.sourceforge.net/packages/make.htm}

	\section{模版文件}\label{sec:doc-dir}

	在正确安装 pkuthss 文档模版之后,在终端/命令提示符中执行
\begin{Verbatim}[frame = single]
texdoc pkuthss
\end{Verbatim}
	所打开的 pdf 文件所在的同一目录中包含三个子目录和两个 pdf 文件,
	其中 \verb|doc-src/| 为说明文档(即本文档)的源代码,
	\verb|example-utf8/| 和 \verb|example-gbk/| 为模版文档的源代码,
	\verb|readme.pdf| 和 \verb|example.pdf|
	分别为说明文档和模版文档的 pdf 文件。

	\verb|example-utf8/| 目录下的源代码使用的是 UTF-8 编码,
	适合类 UNIX 系统用户或编辑器对 UTF-8 有良好支持的用户使用;%
	\verb|example-gbk/| 目录下的源代码使用的是 GBK 编码,适合其它用户使用。%
	用户可以视情况将 \verb|example-utf8/| 或 \verb|example-gbk/|
	中的所有内容复制到合适的目录,
	并在此目录中根据模版修改出自己的论文。

	\myemph{%
		注意:
		在 1.4 rc4 之后版本的 pkuthss 文档模版中,除了 \texttt{Make.bat} 之外,
		所有文件均是 LF(\texttt{\string\n})换行。
		在 Windows 下用“记事本”打开这些文件时,
		所有的换行会变成某个奇怪的字符,而所有文字会挤在一行上。
		这是“记事本”的固有问题,
		因此 pkuthss 文档模版的作者建议用户
		使用支持 LF 换行的文本编辑器编辑文件。%
	}

	模版中的重要文件有:
	\begin{itemize}
		\item \verb|Makefile|:
			被 Make 工具调用的 Makefile,用于使编译工作自动化。
		\item \verb|Make.bat|:%
			Windows 下的伪“Makefile”,用 Windows 批处理实现。

		\item \verb|chap/|:目录,包含各章节内容:
		\begin{itemize}
			\item \verb|copyright.tex|:
				版权声明部分。
			\item \verb|originauth.tex|:
				原创性声明和使用授权说明部分\supercite{pku-originauth}。
		\end{itemize}
		\myemph{%
			注:%
			pkuthss 文档模版支持排版学校要求的二维码,
			请参考 \texttt{copyright.tex} 和
			\texttt{originauth.tex} 中的相关注释。%
		}
	\end{itemize}

	\section{编译方式}\label{sec:compile}

	pkuthss 文档模版支持三种编译方式,即
	\begin{itemize}
		\item \hologo{LaTeX} -- dvipdfmx 方式:\\
			依次执行 \verb|latex|,\verb|biber|(或 \verb|bibtex|),%
			\verb|latex|,\verb|latex| 和 \verb|dvipdfmx|。
		\item \hologo{pdfLaTeX} 方式:\\
			依次执行 \verb|pdflatex|,\verb|biber|(或 \verb|bibtex|),%
			\verb|pdflatex| 和 \verb|pdflatex|。
		\item \hologo{XeLaTeX} 方式:\\
			依次执行 \verb|xelatex|,\verb|biber|(或 \verb|bibtex|),%
			\verb|xelatex| 和 \verb|xelatex|。
	\end{itemize}

	\myemph{%
		注意:
		\begin{itemize}
			\item 除非无法使用 biber,否则不建议使用 bibtex。
				如果的确要使用 bibtex,请将 biblatex 宏包的\texttt{backend}
				选项从 \texttt{biber} 改为 \texttt{bibtex},
				在 Makefile 中将变量 \texttt{BIBTEX} 的值改为 \texttt{bibtex},
				并去掉 biblatex 宏包和 \texttt{\string\printbibliography} 命令
				的 \texttt{sorting} 选项。%
			\item bibtex 对 UTF-8 编码的支持不好,
				因此不建议在编译 UTF-8 编码的源文件时使用 bibtex。
			\item \hologo{XeLaTeX} 对非 UTF-8 编码的支持不好,
				因此 \hologo{XeLaTeX} 方式的编译不支持 GBK 编码。
		\end{itemize}%
	}

	pkuthss 文档模版附带的 Makefile 中已经对这三种编译方式进行了完整的配置。
	用户只需要在 Makefile 中通过设定变量 \verb|JOBNAME| 的值%
	指定被编译的主文件名,
	并通过设定变量 \verb|LATEX| 的值指定采用哪种编译方式,
	即可通过在主文件所在目录调用 Make 工具来实现自动编译:
	如果是在类 UNIX 环境下,则用户应该调用的命令名为 \verb|make|:
\begin{Verbatim}[frame = single]
cd /path/to/directory/with/thesis.tex
make
\end{Verbatim}
	而如果是在 Windows 环境下,
	则用户应该调用的命令名可能为 \verb|mingw32-make|:
\begin{Verbatim}[frame = single]
cd \path\to\directory\with\thesis.tex
mingw32-make
\end{Verbatim}

	用户如果不想配置 Windows 下的 GNU Make,
	则也可以使用由 Windows 批处理实现的伪“Makefile”,
	通过在主文件所在目录调用 \verb|make|\footnote{%
		Windows 将批处理文件作为可执行文件,
		调用时可以不显式地指出扩展名。%
	}:
\begin{Verbatim}[frame = single]
cd \path\to\directory\with\thesis.tex
make
\end{Verbatim}
	\myemph{%
		注意:%
		Windows 批处理对于 LF(\texttt{\string\n})换行的批处理文件支持有问题。
		在命令提示符(cmd)下执行这些批处理文件时没有问题,
		但双击文件图标执行时可能就会出错。
		\hologo{TeX} Live 中安装的 \texttt{Make.bat} 和
		CTAN 上提供的压缩包里的 \texttt{Make.bat}
		有可能是 LF(\texttt{\string\n})换行的。%
	}

