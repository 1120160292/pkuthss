% vim:ts=4:sw=4
%
% Documentation for pkuthss.
%
% Copyright (c) 2008-2009 solvethis
% Copyright (c) 2010-2012,2014 Casper Ti. Vector
%
% This work may be distributed and/or modified under the conditions of the
% LaTeX Project Public License, either version 1.3 of this license or (at
% your option) any later version.
% The latest version of this license is in
%   http://www.latex-project.org/lppl.txt
% and version 1.3 or later is part of all distributions of LaTeX version
% 2005/12/01 or later.
%
% This work has the LPPL maintenance status `maintained'.
% The current maintainer of this work is Casper Ti. Vector.
%
% This work consists of the following files:
%   pkuthss.tex
%   chap/copyright.tex
%   chap/abstract.tex
%   chap/introduction.tex
%   chap/chap1.tex
%   chap/chap2.tex
%   chap/chap3.tex
%   chap/conclusion.tex
%   chap/encl1.tex
%   chap/encl2.tex
%   chap/acknowledge.tex

\chapter{pkuthss 文档模版的实现}
\raggedbottom

	\section{pkuthss 文档类和 pkuthss-extra 宏包的实现}	
		\subsection{共用文件头部}
		\VerbatimInput[
			frame = lines, fontsize = {\footnotesize}, tabsize = 2,
			baselinestretch = 1, lastline = 23, numbers = left
		]{pkuthss.cls}

		\subsection{\texttt{pkuthss.cls}}
		\VerbatimInput[
			frame = lines, fontsize = {\footnotesize}, tabsize = 2,
			baselinestretch = 1, firstline = 25, numbers = left
		]{pkuthss.cls}

		\subsection{\texttt{pkuthss-utf8.def} 和 \texttt{pkuthss-gbk.def}}
		\VerbatimInput[
			frame = lines, fontsize = {\footnotesize}, tabsize = 2,
			baselinestretch = 1, firstline = 25, numbers = left
		]{pkuthss-utf8.def}

		\subsection{\texttt{pkuthss-extra.sty}}
		\VerbatimInput[
			frame = lines, fontsize = {\footnotesize}, tabsize = 2,
			baselinestretch = 1, firstline = 25, numbers = left
		]{pkuthss-extra.sty}

	\section{pkuthss 说明(示例)文档的源代码}

	本文档的源代码中大部分已经有了较为详细的注释,
	故请直接参照相应文件中的注释。

	\myemph{%
		注:%
		\texttt{img/} 目录中的 \texttt{Makefile} 和%
		两个 PostScript(\texttt{.eps})文件(都是文本文件)中%
		也有详细的注释哦 :)%
	}

\flushbottom

