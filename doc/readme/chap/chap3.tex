% vim:ts=4:sw=4
%
% Documentation for pkuthss.
%
% Copyright (c) 2008-2009 solvethis
% Copyright (c) 2010-2014 Casper Ti. Vector
%
% This work may be distributed and/or modified under the conditions of the
% LaTeX Project Public License, either version 1.3 of this license or (at
% your option) any later version.
% The latest version of this license is in
%   http://www.latex-project.org/lppl.txt
% and version 1.3 or later is part of all distributions of LaTeX version
% 2005/12/01 or later.
%
% This work has the LPPL maintenance status `maintained'.
% The current maintainer of this work is Casper Ti. Vector.
%
% This work consists of the following files:
%   pkuthss.tex
%   chap/copyright.tex
%   chap/abstract.tex
%   chap/introduction.tex
%   chap/chap1.tex
%   chap/chap2.tex
%   chap/chap3.tex
%   chap/conclusion.tex
%   chap/encl1.tex
%   chap/acknowledge.tex

\chapter{问题及其解决}
	\section{文档中已经提到的常见问题(按重要性排序)}

	北大图书馆要求上传的论文中目录采用黑色字体;
	在默认设置(启用 \verb|colorlinks| 选项)下,
	黑白打印时文档中的部分彩色链接可能会变成浅灰色,
	解决方式见第 \ref{ssec:extra} 小节。

	中文字体字库不全(只包含 GB2312 字符集内字符)时,
	生成的 pdf 文档中可能缺少部分字符,
	解决方式见第 \ref{sec:req} 节。

	Windows 批处理对于 LF(\texttt{\string\n})换行的批处理文件支持有问题,
	解决方式见第 \ref{sec:compile} 节。

	Windows 的“记事本”程序在查看 LF(\texttt{\string\n})
	换行的文本文件时存在着一些问题,
	因此建议用户使用支持 LF 换行的文本编辑器编辑文件,
	详见第 \ref{sec:doc-dir} 节。

	使用过旧的 \TeX{} 系统和各宏包,
	或使用某些 Linux 发行版软件仓库所提供的 \TeX{}Live 时,
	可能引起一些问题,
	详见第 \ref{sec:req} 节。

	文档默认情况下是双面模式,章末可能产生空白页,详见第 \ref{ssec:options} 小节。

	通过一些设置,还可以满足例如被引用的文献按照引用顺序排序,
	而未引用的文献按照英文文献在前、中文文献在后排序这样的需求,
	见第 \ref{ssec:thirdparty} 小节。

	无法使用 biber 的用户可以使用 bibtex,这种情况下建议使用 GBK 编码,
	具体设置方式请见第 \ref{sec:compile} 节。

	一些高级设置,
	如封面中部分内容长度超过预设空间容量时的设置,
	见第 \ref{sec:advanced} 节。

	\section{其它可能存在的问题}
		\subsection{上游宏包可能引起的问题}

		biblatex 宏包\supercite{biblatex}会自行设定 \verb|\bibname|,
		故会覆盖通过 \verb|\CTEXoptions| 设定的参考文献列表标题。
		使用 biblatex 的用户可以使用 \verb|\printbibliography| 的
		\verb|title| 选项来手动设定参考文献列表的标题,例如:
\begin{Verbatim}[frame = single]
\printbibliography[title = {文献}, ...] % “...”为其它选项。
\end{Verbatim}

		hyperref 宏包\supercite{hyperref}和一些宏包可能发生冲突。
		关于如何避免这些冲突,
		可以参考 hyperref 宏包的 README 文件中的“Package Compatibility”一节。
		此文件通常和执行 \verb|texdoc hyperref| %
		时打开的 pdf 文件位于同一目录中。

		使用 Xe\LaTeX{} 的用户可能在已经安装字体的情况下遇到形如(其中
		\verb|xxxxxxxx| 为具体字体名)
\begin{Verbatim}[frame = single, fontsize = {\small}]
! fontspec error: "font-not-found"
! The font "xxxxxxxx" cannot be found.
! See the fontspec documentation for further information.
! For immediate help type H <return>.
\end{Verbatim}
		的错误。
		这种错误一般是(主要是非 Windows 平台的)用户采用了自定义的(包括大小写
		不同于原文件的)字体文件名,
		并改动 ctex.cfg 等配置文件之后没有在调用 pkuthss 文档类时加入
		\verb|nofonts| 选项,
		又使用 Xe\LaTeX{} 编译造成的,使用
\begin{Verbatim}[frame = single]
\documentclass[nofonts, ...]{pkuthss} % “...”代表其它的选项。
\end{Verbatim}
		即可解决此问题。

		biber 运行时有一定概率出现形如(目录名可能稍有不同)
\begin{Verbatim}[frame = single, fontsize = {\small}]
data source .../par-xxxxxxxx/cache-xxxxxxxx/
	inc/lib/Biber/LaTeX/recode_data.xml not found in .
\end{Verbatim}
		的错误。
		这种错误一般是 biber 在自解压阶段被终止之后,
		未删除临时目录 \verb|.../par-xxxxxxxx/| 就重新运行 biber 时出现。
		遇到这种情况时,删除掉上述临时目录及其所有内容,
		再重新运行 biber 通常便可解决问题。

		caption 宏包\supercite{caption}对于其不认识的宏包均会提示
\begin{Verbatim}[frame = single, fontsize = {\small}]
Package caption Warning: Unsupported document class (or package) detected,
(caption)                usage of the caption package is not recommended.
See the caption package documentation for explanation.
\end{Verbatim}
		pkuthss 文档模版基于 ctexbook 文档类,
		而后者基于标准的 book 文档类。
		因此,这个警告并不影响用户正常使用\footnote{%
			\url{http://bbs.ctex.org/forum.php?mod=redirect&goto=findpost&ptid=63117&pid=402145}.%
		}。

		\subsection{文档格式可能存在的问题}

		研究生手册和其电子版\supercite{pku-thesisstyle}要求的论文封面可能并不一致,
		这里以电子版为准。

	\section{反馈意见和建议}

	关于 pkuthss 文档模版的意见和建议,
	请在北大未名 BBS 的 MathTools 版或 %
	Google Code 上 pkuthss 项目的 issue tracker%
	\footnote{\url{https://code.google.com/p/caspervector/issues/list}.}%
	上提出,
	或通过电子邮件\footnote%
	{\href{mailto:CasperVector@gmail.com}{\texttt{CasperVector@gmail.com}}.}%
	告知 Casper Ti. Vector。
	上述三种反馈方法中,建议用户尽量采用靠前的方法。

	在进行反馈时,请尽量确保已经仔细阅读本文档中的说明。
	如果是通过 BBS 或电子邮件进行反馈,
	请在标题中说明是关于 pkuthss 文档模版的反馈;
	如果是通过 Google Code 进行反馈,
	请给 issue 加上 \verb|Proj-Pkuthss| 标签。
	如果是错误报告,
	请说明所使用 pkuthss 模版的版本、
	自己使用的操作系统和 \TeX{} 系统的类型和版本;
	同时强烈建议附上一个出错的最小例子及其相应的编译日志(\verb|.log| 文件),
	在文件较长时请使用附件。

