% vim:ts=4:sw=4
%
% Copyright (c) 2008-2009 solvethis
% Copyright (c) 2010-2012 Casper Ti. Vector
% Public domain.

\chapter{pkuthss 文档模版提供的功能}
	\section{pkuthss 文档模版提供的文档类和宏包选项}
		\subsection{pkuthss 文档类提供的选项}\label{ssec:options}

		\begin{itemize}
			\item \textbf{\texttt{[no]extra}}:
				用于确定是否自动载入 pkuthss-extra 宏包。
				在默认情况下,pkuthss 文档类将使用 \verb|extra| 选项。
				用户如果不需要自动载入 pkuthss-extra 宏包,
				则需要在载入 pkuthss 时加上 \verb|noextra| 选项。

			\item \textbf{pkuthss-extra 宏包提供的选项}:
				这些选项将被传递给 pkuthss-extra 宏包
				(用户需要启用 \verb|extra| 选项)。
				具体说明参见第 \ref{ssec:extra} 小节。

			\item \textbf{其余文档类选项}:%
				pkuthss 文档类以 ctexbook 文档类为基础,
				其接受的其余所有文档类选项均被传递给 ctexbook。
				其中可能最常用的选项是 \verb|GBK| 和 \verb|UTF8|:
				它们选择源代码使用的字符编码,默认使用 \verb|GBK|。
		\end{itemize}

		例如,如果需要使用 UTF-8 编码撰写论文,
		则需要在导入 pkuthss 文档类时加上 \verb|UTF8| 选项:
\begin{Verbatim}[frame = single]
\documentclass[UTF8, ...]{pkuthss} % “...”代表其它的选项。
\end{Verbatim}

		又例如,文档默认情况下是双面模式,每章都从右页(奇数页)开始。
		如果希望改成一章可以从任意页开始,可以这样设置:
\begin{Verbatim}[frame = single]
\documentclass[openany, ...]{pkuthss} % 每章从任意页开始。
\end{Verbatim}
		但这样设置时左右(奇偶)页的页眉页脚设置仍然是不同的。
		如果需要使左右页的页眉页脚设置一致,可以直接采用单面模式:
\begin{Verbatim}[frame = single]
% 使用 oneside 选项时不需要再指定 openany 选项。
\documentclass[oneside, ...]{pkuthss}
\end{Verbatim}

		\subsection{pkuthss-extra 宏包提供的选项}\label{ssec:extra}

		除非特别说明,
		下面提到的选项中都是不带“\verb|no|”的版本被启用。

		\begin{itemize}
			\item \textbf{\texttt{[no]spacing}}:
				是否采用一些常用的对空白进行调整的版式设定。
				具体地说,启用 \verb|space| 选项后会进行以下几项设置:
			\begin{itemize}
				\item 自动忽略 CJK 文字之间的空白而%
					保留(CJK 文字与英文之间等的)其它空白。
				\item 调用 setspace 宏包以使某些细节处的空间安排更美观。
				\item 设置页芯居中。
				\item 设定行距为 1.41\footnote{%
					为什么是 1.41?因为 $\sqrt{2}\approx1.41$。%
				}。
				\item 使脚注编号和脚注文本之间默认间隔一个空格。
			\end{itemize}

			\item \textbf{\texttt{[no]tightlist}}:
				是否采用比 \LaTeX{} 默认设定更加紧密的枚举环境。
				在枚举环境(itemize、enumerate 和 description)中,
				每个条目的内容较少时,条目往往显得稀疏;
				在参考文献列表中也有类似的现象。
				启用 \verb|tightlist| 选项后,
				将去掉这些环境中额外增加的(垂直)间隔。

			\item \textbf{\texttt{[no]pdftoc}}\footnote{%
					此选项部分等价于 1.4 alpha2 及以前版本 pkuthss-extra 宏包%
					的 \texttt{[no]tocbibind} 选项。
					因为 tocbibind 宏包和 biblatex 宏包冲突,
					pkuthss-extra 宏包不再调用 tocbibind 宏包。%
				}:
				启用 \verb|pdftoc| 选项后,
				用 \verb|\tableofcontents| 命令生成目录时%
				会自动添加“目录”的 pdf 书签。

			\item \textbf{\texttt{[no]uppermark}}:
				是否在页眉中将章节名中的小写字母转换为大写字母。
				就目前而言,
				这样的转换存在着一些较为严重的缺陷\footnote{%
					准确地说是 \texttt{\string\MakeUppercase} 宏的问题,
					其在某些地方的转换不够健壮,
					例如 \texttt{\string\cite\string{ctex\string}}
					会被转换成 \texttt{\string\cite\string{CTEX\string}}。%
				},
				因此不建议使用。
				基于上述考虑,%
				\myemph{%
					pkuthss-extra 宏包默认启用 \texttt{nouppermark} 选项,
					即在不在页眉中使用大写的章节名%
				}。

			\item \textbf{\texttt{[no]spechap}}\footnote{%
					“spechap”是“\textbf{spec}ial \textbf{chap}ter”的缩写。%
				}:
				是否启用第 \ref{ssec:misc} 小节中介绍的 %
				\verb|\specialchap| 命令。

			\item \textbf{\texttt{[no]pdfprop}}:
				是否自动根据设定的论文文档信息(如作者、标题等)
				设置生成的 pdf 文档的相应属性。%
				\myemph{%
					注意:
					该选项实际上是在 \texttt{\string\maketitle} 时生效的,
					这是因为考虑到%
					通常用户在调用 \texttt{\string\maketitle} 前%
					已经设置好所有的文档信息。
					若用户不调用 \texttt{\string\maketitle},
					则需在设定完文档信息之后自行调用%
					第 \ref{ssec:misc} 小节中介绍的 %
					\texttt{\string\setpdfproperties} 命令以完成%
					pdf 文档属性的设定。%
				}

			\item \textbf{\texttt{[no]colorlinks}}\footnote{%
					此选项等价于 1.3 及以前版本 pkuthss-extra 宏包%
					的 \texttt{[no]linkcolor} 选项,
					但后来发现这会和 hyperref 宏包的一个同名选项冲突,
					故改为 \texttt{[no]colorlinks}。%
				}:
				是否在生成的 pdf 文档中使用彩色的链接。
		\end{itemize}

		例如,在提交打印版的论文时,
		彩色的链接文字在黑白打印出来之后可能颜色会很浅。
		此时用户\myemph{%
			可以启用 pkuthss-extra 宏包的 nocolorlinks 选项,
			使所有的链接变为黑色,以免影响打印%
		}:
\begin{Verbatim}[frame = single]
\documentclass[..., nocolorlinks]{pkuthss} % “...”代表其它的选项。
\end{Verbatim}
		用户还可以进一步修改 \verb|img/| 目录中 %
		\verb|pkulogo.eps| 和 \verb|pkuword.eps| 两个图片文件,
		以使封面上的北京大学图样也变为黑色(详见相应文件中的注释)。

	\section{pkuthss 文档模版提供的命令和环境}
		\subsection{设定文档信息的命令}

		这一类命令的语法为
\begin{Verbatim}[frame = single]
\commandname{具体信息} % commandname 为具体命令的名称。
\end{Verbatim}

		这些命令总结如下:
		\begin{itemize}
			\item \texttt{\bfseries\string\ctitle}:设定论文中文标题;
			\item \texttt{\bfseries\string\etitle}:设定论文英文标题;
			\item \texttt{\bfseries\string\cauthor}:设定作者的中文名;
			\item \texttt{\bfseries\string\eauthor}:设定作者的英文名;
			\item \texttt{\bfseries\string\studentid}:设定作者的学号;
			\item \texttt{\bfseries\string\date}:设定日期;
			\item \texttt{\bfseries\string\school}:设定作者的学院名;
			\item \texttt{\bfseries\string\cmajor}:设定作者专业的中文名;
			\item \texttt{\bfseries\string\emajor}:设定作者专业的英文名;
			\item \texttt{\bfseries\string\direction}:设定作者的研究方向;
			\item \texttt{\bfseries\string\cmentor}:设定导师的中文名;
			\item \texttt{\bfseries\string\ementor}:设定导师的英文名;
			\item \texttt{\bfseries\string\ckeywords}:设定中文关键词;
			\item \texttt{\bfseries\string\ekeywords}:设定英文关键词。
		\end{itemize}

		例如,如果要设定专业为“化学”(“Chemistry”),则可以使用以下命令:
\begin{Verbatim}[frame = single]
\cmajor{化学}
\emajor{Chemistry}
\end{Verbatim}

		\subsection{自身存储文档信息的命令}

		这一类命令的语法为
\begin{Verbatim}[frame = single]
% commandname 为具体的命令名。
\renewcommand{\commandname}{具体信息}
\end{Verbatim}

		这些命令总结如下:
		\begin{itemize}
			\item \texttt{\bfseries\string\cuniversity}:大学的中文名。
			\item \texttt{\bfseries\string\euniversity}:大学的英文名。
			\item \texttt{\bfseries\string\cthesisname}:论文类别的中文名。
			\item \texttt{\bfseries\string\ethesisname}:论文类别的英文名。
			\item \texttt{\bfseries\string\cabstractname}:摘要的中文标题。
			\item \texttt{\bfseries\string\eabstractname}:摘要的英文标题。
		\end{itemize}

		例如,
		如果要设定论文的类别为“本科生毕业论文”(“Undergraduate Thesis”),
		则可以使用以下命令:
\begin{Verbatim}[frame = single]
\renewcommand{\cthesisname}{本科生毕业论文}
\renewcommand{\ethesisname}{Undergraduate Thesis}
\end{Verbatim}

		\subsection{以“key = value”格式设置文档信息}

		用户可以通过 \verb|\pkuthssinfo| 命令集中设定文档信息,
		其语法为:
\begin{Verbatim}[frame = single]
% key1、key2、value1、value2 等为具体文档信息的项目名和内容。
\pkuthssinfo{key1 = value1, key2 = value2, ...}
\end{Verbatim}
		其中文档信息的项目名为前面提到的设定文档信息的命令名%
		或自身存储文档信息的命令名(不带反斜杠)。

		当文档信息的内容包含了逗号等有干扰的字符时,
		可以用大括号将这一项文档信息的全部内容括起来。%
		\myemph{%
			我们推荐用户总用大括号将文档信息的内容括起来,
			以避免很多不必要的麻烦。%
		}

		例如,前面提到的文档信息的设置可以集中地写成:
\begin{Verbatim}[frame = single, tabsize = 4]
\pkuthssinfo{
	..., % “...”代表其它的设定。
	cthesisname = {本科生毕业论文},
	ethesisname = {Undergraduate Thesis},
	cmajor = {化学}, emajor = {Chemistry}
}
\end{Verbatim}

		\subsection{pkuthss 文档模版提供的其它命令和环境}\label{ssec:misc}

		\texttt{\bfseries cabstract} 和 \texttt{\bfseries eabstract} %
		环境用于编写中英文摘要。
		用户只需要写摘要的正文;标题、作者、导师、专业等部分会自动生成。

		\texttt{\bfseries\string\specialchap} 命令%
		用于开始不进行标号但计入目录的一章,
		并合理安排其页眉。%
		\myemph{%
			注意:
			需要启用 pkuthss-extra 宏包的 \texttt{spechap} 选项%
			才能使用此命令。
			另外,在此章内的节或小节等命令应使用带星号的版本,
			例如 \texttt{\string\section\string*} 等,
			以免造成章节编号混乱。%
		}%
		例如,本文档中的“绪言”一章就是用 \verb|\specialchap{绪言}| %
		这条命令开始的。%

		\texttt{\bfseries\string\setpdfproperties} 命令%
		用于根据用户设定的文档信息自动设定生成的 pdf 文档的属性。
		此命令会在用户调用 \verb|\maketitle| 命令时被自动调用,
		因此通常不需要用户自己使用;
		但用户有时可能不需要输出标题页,
		从而不会调用 \verb|\maketitle| 命令,
		此时就需要在设定完文档信息之后调用 \verb|\setpdfproperties|。
		\myemph{%
			注意:
			需要启用 pkuthss-extra 宏包的 \texttt{pdfprop} 选项%
			才能使用此命令。%
		}

		\subsection{从其它文档类和宏包继承的功能}\label{ssec:thirdparty}

		pkuthss 文档类建立在 ctexbook\supercite{ctex} 文档类的基础上,
		并调用了 CJKfntef、%
		graphicx\supercite{graphicx}、geometry\supercite{geometry}、%
		fancyhdr\supercite{fancyhdr} 和 %
		etoolbox\supercite{etoolbox} 等几个宏包。
		因此,ctexbook 文档类和这些宏包所提供的功能均可以使用。

		例如,用户如果想将目录的标题改为“目{\quad\quad}录”,
		则可以使用 ctexbook 文档类提供的 \verb|\CTEXoptions| 命令:
\begin{Verbatim}[frame = single]
\CTEXoptions{contentsname = {目{\quad\quad}录}}
\end{Verbatim}

		在默认的配置下,%
		pkuthss 文档模版使用作者编写的 %
		biblatex\supercite{biblatex} 样式\supercite{biblatex-caspervector}%
		进行参考文献和引用的排版,
		用户可以使用它以及 biblatex 本身所提供的功能。
		例如,
		用户可以分别使用 \verb|\cite|、\verb|\parencite| 和 \verb|\supercite| %
		生成未格式化的、带方括号的和上标且带方括号的引用标记:
\begin{Verbatim}[frame = single]
\cite{ctex},\parencite{ctex},\supercite{ctex}
\end{Verbatim}
		在本文中将产生“\cite{ctex},\parencite{ctex},\supercite{ctex}”。

		通过更复杂的设置,还可以满足例如被引用的文献按照引用顺序排序,
		而未引用的文献按照英文文献在前、中文文献在后排序这样的需求,
		详见 biblatex-caspervector 的文档\supercite{biblatex-caspervector}。

		pkuthss-extra 宏包可能调用以下这些宏包:
		\begin{itemize}
			\item 启用 \verb|spacing| 选项时会调用 %
				setspace 和 footmisc\supercite{footmisc} 宏包。
			\item 启用 \verb|tightlist| 选项时会调用 %
				enumitem\supercite{enumitem} 宏包。
		\end{itemize}
		因此在启用相应选项时,用户可以使用对应宏包所提供的功能。

		\subsection{不建议更改的设置}
		\myemph{%
			pkuthss 文档类中有一些一旦改动就有可能破坏预设排版规划的设置,
			因此不建议更改这些设置,它们是:
			\begin{itemize}
				\item 纸张类型:A4;
				\item 页芯尺寸:%
					$240\,\mathrm{mm}\times150\,\mathrm{mm}$,
					包含页眉、页脚;
				\item 默认字号:小四号。
			\end{itemize}%
		}

	\section{高级设置}\label{sec:advanced}

	pkuthss 文档模版的实现是简洁、清晰、灵活的。
	当一些细节的自定义无法通过模版提供的外部接口实现时,
	我们鼓励用户(在适当理解相关部分代码的前提下)通过修改模版进行自定义。

	一个常见的需求是封面中部分内容(特别是论文的标题、专业和研究方向)太长,
	超出了在预设的空间。
	此时,
	用户可以修改 \verb|pkuthss.cls| 里 \verb|\maketitle| 定义中
	\verb|\pkuthss@int@fillinblank| 宏的参数来改变
	带下划线的空白的行数和行宽,其语法为:
\begin{Verbatim}[frame = single]
\pkuthss@int@fillinblank{行数}{行宽}{内容}
\end{Verbatim}
	例如,如果“研究方向”一栏需要两行的空白,
	可以将 \verb|pkuthss.cls| 里的
\begin{Verbatim}[frame = single]
\pkuthss@int@fillinblank{1}{\pkuthss@tmp@len}{\kaishu\@direction}
\end{Verbatim}
	改为
\begin{Verbatim}[frame = single]
\pkuthss@int@fillinblank{2}{\pkuthss@tmp@len}{\kaishu\@direction}
\end{Verbatim}
	当然,为了美观,可以将多于一行的部分移到封面中作者信息部分的最下方。

