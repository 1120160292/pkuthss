% Copyright (c) 2008-2009 solvethis
% Copyright (c) 2010-2011 Casper Ti. Vector
% Public domain.

\specialchap{绪言}

本文档是“北京大学论文文档模版”的说明文档,
同时也是使用模版的一个示例。

pkuthss 文档模版由三部分构成:
\begin{itemize}
	\item \textbf{pkuthss 文档类}:
		其中进行了学位论文所需要的一些基本的设定,
		主要包括对基本排版格式的设定和提供设置论文信息的命令。
	\item \textbf{pkuthss-extra 宏包}:
		其中实现了学位论文中用户可能较多用到的一些额外功能,
		例如自动在目录中加入参考文献和索引的条目和%
		自动根据用户设定的文档信息对所生成 pdf 的作者、标题等属性进行设置等。
	\item \textbf{说明(示例)文档}:
		说明文档即本文档,
		在非“绿色”安装(见第 \ref{sec:inst} 节)之后应该可以%
		用 \TeX{} 系统提供的 \verb|texdoc| 命令调出:
\begin{Verbatim}[frame=single]
texdoc pkuthss
\end{Verbatim}
		同时,
		本文档的源代码(位于 \verb|doc/| 目录下)%
		也正是用户撰写自己的学位论文时的一个模版:
		用户只需按照模版中的框架修改代码,
		即可写出自己的论文。
\end{itemize}

在此之前,包括 dypang\cite{dypang}、FerretL\cite{FerretL}、%
lwolf\cite{lwolf}、Langpku\cite{Langpku}、%
solvethis\cite{solvethis} 等的数位网友均做过学位论文模版的工作。
本论文模版是 solvethis 的 pkuthss 模版的更新版本,
更新的重点是重构和对新文档类、宏包的支持。

pkuthss 文档模版现在的维护者是 Casper Ti. Vector\footnote%
{\href{mailto:CasperVector@gmail.com}{\texttt{CasperVector@gmail.com}}}。%
pkuthss 文档模版目前托管在 Google Code 上,
其项目主页是:\\
\hspace*{\parindent}\url{http://code.google.com/p/caspervector/}

