% vim:ts=4:sw=4
%
% Copyright (c) 2008-2009 solvethis
% Copyright (c) 2010-2012 Casper Ti. Vector
% Public domain.

\chapter{使用介绍}
	\section{重要文件}
	\subsection{\texttt{tex/} 目录}

	本目录中包含了 pkuthss 文档类和 pkuthss-extra 宏包:

	\begin{itemize}
		\item \verb|pkuthss.cls|:%
			pkuthss 文档类的主文件。
		\item \verb|pkuthss-utf8.def|:%
			pkuthss 文档类中使用的一些标题和标记文字,使用 UTF-8 编码。
		\item \verb|pkuthss-gbk.def|:%
			pkuthss 文档类中使用的一些标题和标记文字,使用 GBK 编码。
		\item \verb|pkuthss-extra.sty|:
			在论文中一些常见的额外设置。
	\end{itemize}
	
	\subsection{\texttt{doc/} 目录}

	本目录中包含本文档(\verb|README.pdf|)的源代码%
	(\verb|utf8lf/| 和 \verb|gbkcrlf/| 两个子目录,
	两目录中代码除使用的编码和换行符外基本上完全相同),
	以及 pkuthss 文档模版所使用的许可证(\verb|license/| 目录)。

	其中,%
	\verb|utf8lf/| 目录下的源代码使用的是 UTF-8 编码、\verb|\n|(LF)换行,
	适合类 UNIX 系统用户使用;%
	\verb|gbkcrlf/| 目录下的源代码使用的是 GBK 编码、\verb|\r\n|(CRLF)换行,
	适合 Windows 用户使用。%
	\verb|utf8lf/| 和 \verb|gbkcrlf/| 目录中的重要文件有:
	\begin{itemize}
		\item \verb|Makefile|:
			被 Make 工具调用的 Makefile,用于使编译工作自动化。
		\item \verb|Make.bat|:%
			Windows 下的伪“Makefile”,用 Windows 批处理实现。
		\item \verb|chap/|:目录,包含各章节内容:
		\begin{itemize}
			\item \verb|copyright.tex|:
				版权声明部分\footnote{%
					本文档中的版权声明并不是%
					学校默认要求的形式\cite{pku-copyright}。
					符合学校要求的一个版权声明已经放在此文件中,
					但用 \texttt{\string\iffalse{} ...\ \string\fi} %
					注释掉了,
					用户可以考虑使用那个版本。
				}。
			\item \verb|originauth.tex|:
				原创性声明和使用授权说明部分\cite{pku-originauth}。
		\end{itemize}
		\item \verb|img/|:目录,包含论文中所有图片:
		\begin{itemize}
			\item \verb|Makefile|:图片部分的 Makefile。
			\item \verb|Make.bat|:%
				Windows 下的伪“Makefile”,用 Windows 批处理实现。
			\item \verb|pkulogo.eps|:北京大学校徽。
			\item \verb|pkuword.eps|:“北京大学”字样。
		\end{itemize}
	\end{itemize}

	\section{系统要求}

	正确编译需要以下几部分:
	\begin{itemize}
		\item 一个基本的 \TeX{} 发行版。
		\item CJK 或 XeCJK(供 Xe\LaTeX{} 使用)宏包。
		\item ctex 宏包\cite{ctex,ctex-faq}(提供了 ctexbook 文档类)。
		\item 中文字体\footnote{\emph{%
			一些中文字体的字库不全,
			只有 GB2312 字符集内字符的字体信息。
			这种情况通常会造成编译生成的 pdf 文件中缺少部分字符,
			其中一种典型症状是封面的“〇”字显示不出来。
			如果要使用中易公司的字体,
			则建议使用 Windows Vista 及其以后版本提供的%
			宋体、黑体、楷体和仿宋体,
			以及 Microsoft Office 2003 及其以后版本提供的隶书和幼圆体,
			这些字体是 GB18030 字符集的,不存在上述问题。%
		}}。
		\item 如果需要使用 Makefile 来实现自动编译,还需要 Make 工具;
			但如果使用由批处理实现的伪“Makefile”就不用了。
	\end{itemize}

	最新的 \TeX{}Live 系统和 \CTeX{} 套装都已经包含%
	除中文字体之外所有要求的项目;
	中文字体需要用户自行获得。

	Linux 用户可以从软件源获得 GNU 的 make;
	其它类 UNIX 系统应该也会提供 make 工具,请参阅相应的文档以获得帮助。%
	Windows 用户可以从以下地址下载 Windows 下的 GNU make 工具:\\
	\hspace*{\parindent}%
	\url{http://gnuwin32.sourceforge.net/packages/make.htm}

	为了获得最好的支持,
	我们建议用户使用最新版的 \TeX{} 系统和各宏包。
	根据目前已有的测试结果,
	使用 \TeX{}Live 2009 及其以上稳定版本%
	或 \CTeX{} 套装 v2.8 系列及其以上版本
	的用户可以正常使用本模版。
	
	\section{安装}\label{sec:inst}
	\subsection{“绿色”安装}

	将 \verb|tex/| 中的所有文件复制到 %
	\verb|doc/| 下合适的编码和换行符所对应的目录%
	(\verb|utf8lf/| 或 \verb|gbkcrlf/|)中即可。

	\subsection{非“绿色”安装}

	对于支持 \TeX{} 目录结构(TDS)标准的 \TeX{} 发行版%
	(其中包括 \TeX{}Live 和 \CTeX{}),
	安装 pkuthss 文档模版都可以通过类似的方式:
	将 \verb|tex/| 中内容放在 \verb|$TEXMF/tex/latex/pkuthss/|,
	将本文档(\verb|README.pdf|)和 \verb|doc/| 中内容%
	放在 \verb|$TEXMF/doc/latex/pkuthss/|。
	其中 \verb|$TEXMF| 是用户系统上的 texmf 目录,
	我们推荐使用 \verb|$TEXMFHOME| 或者 \verb|$TEXMFLOCAL|:
	\begin{itemize}
		\item 在 Windows 下,%
			\TeX{}Live 和 \CTeX{} 套装默认的 \verb|$TEXMFHOME| %
			通常可能都是
\begin{Verbatim}[frame = single]
# user 也可改为其它名字,依你的用户名而定。
C:\Documents and Settings\user\texmf
\end{Verbatim}
			而 \verb|$TEXMFLOCAL| 通常可能分别是
\begin{Verbatim}[frame = single]
# C:\Program Files\CTEX 也可改为其它目录,依安装位置而定。
C:\Program Files\CTEX\MiKTeX\texmf-local
\end{Verbatim}
			和
\begin{Verbatim}[frame = single]
# 2010 也可改为其它年份,依 TeXLive 版本而定。
C:\Program Files\texlive\2010\texmf-local
\end{Verbatim}
		\item 在 Linux 和其它类 UNIX 操作系统下,
			\TeX{}Live 默认的 \verb|$TEXMFHOME| 通常可能是
\begin{Verbatim}[frame = single]
$HOME/texmf
\end{Verbatim}
			而 \verb|$TEXMFLOCAL| 通常可能是
\begin{Verbatim}[frame = single]
# 2010 也可改为其它年份,依 TeXLive 版本而定。
/usr/local/texlive/2010/texmf-local
\end{Verbatim}
	\end{itemize}

	用这样的方法安装完宏包后,可能需要刷新一次文件名数据库。
	关于这一步如何操作,详见 \CTeX{} FAQ\cite{ctex-faq}\footnote{%
		在类 UNIX 操作系统的终端或 Windows 的命令提示符中%
		用 \texttt{texdoc ctex-faq} 调出文档,
		然后在其中搜索“文件名数据库”即可找到。%
	}。

	\section{编译方式}

	pkuthss 文档模版支持三种编译方式,即
	\begin{itemize}
		\item \LaTeX{} -- dvipdfmx 方式:\\
			依次执行 \verb|latex|,\verb|bibtex|,%
			\verb|latex|,\verb|latex| 和 \verb|dvipdfmx|。
		\item pdf\LaTeX{} 方式:\\
			依次执行 \verb|pdflatex|,\verb|bibtex|,%
			\verb|pdflatex| 和 \verb|pdflatex|。
		\item Xe\LaTeX{} 方式:\\
			依次执行 \verb|xelatex|,\verb|bibtex|,%
			\verb|xelatex| 和 \verb|xelatex|。\\
			\emph{%
				注意:Xe\LaTeX{} 对非 UTF-8 的编码支持不好,
				因此 Xe\LaTeX{} 方式的编译不支持 GBK 编码。
			}
	\end{itemize}

	pkuthss 文档模版附带的 Makefile 中已经对这三种编译方式进行了完整的配置。
	用户只需要在 Makefile 中通过设定变量 \verb|JOBNAME| 的值%
	指定被编译的主文件名,
	并通过设定变量 \verb|LATEX| 的值指定采用哪种编译方式,
	即可通过在主文件所在目录调用 Make 工具来实现自动编译:
	如果是在类 UNIX 环境下,则用户应该调用的命令名为 \verb|make|;
	而如果是在 Windows 环境下,
	则用户应该调用的命令名可能为 \verb|mingw32-make|。

	用户如果不想配置 Windows 下的 GNU Make,
	则也可以使用由 Windows 批处理实现的伪“Makefile”,
	通过在主文件所在目录调用 \verb|make|\footnote{%
		Windows 将批处理文件作为可执行文件,
		调用时可以不显式地指出扩展名。%
	} 或直接双击 \verb|Make.bat| 的图标运行之。

