% vim:ts=4:sw=4
%
% Copyright (c) 2008-2009 solvethis
% Copyright (c) 2010-2013 Casper Ti. Vector
% Public domain.

\chapter{使用介绍}
	\section{系统要求}\label{sec:req}

	正确编译需要以下几部分:
	\begin{itemize}
		\item 一个基本的 \TeX{} 发行版。
		\item CJK 或 XeCJK(供 Xe\LaTeX{} 使用)宏包。
		\item ctex 宏包\supercite{ctex,ctex-faq}(提供了 ctexbook 文档类)。
		\item 中文字体。
		\item 如果要使用 biblatex 进行文献列表和引用的排版的话,
			还需要 biblatex 宏包\supercite{biblatex};
		\item 如果需要对中文文献进行按汉语拼音的排序的话,
			还需要 biber 程序\supercite{biber}。
		\item 如果默认的文献列表和引用样式的话,
			还需要作者编写的 biblatex 样式
			(biblatex-caspervector)\supercite{biblatex-caspervector}。
		\item 如果需要使用 Makefile 来实现自动编译,还需要 Make 工具;
			但如果使用由批处理实现的伪“Makefile”就不用了。
	\end{itemize}

	\myemph{最新}的\myemph{完全版} \TeX{}Live 系统(\myemph{%
		注:
		某些 Linux 发行版软件仓库中的 \TeX{}Live 有问题,
		建议使用独立安装版的 \TeX{}Live%
		\footnote{\url{http://www.tug.org/texlive/}.}。%
	})和 \CTeX{} 套装都已经包含除中文字体和 Make 之外所有要求的项目。%
	\myemph{%
		为了获得最好的支持,
		我们建议用户使用最新完全版的 \TeX{} 系统和各宏包。%
	}

	中文字体需要用户自行获得。\myemph{%
		注:
		一些中文字体的字库不全,
		只有 GB2312 字符集内字符的字体信息。
		这种情况通常会造成编译生成的 pdf 文件中缺少部分字符,
		其中一种典型症状是封面的“〇”字显示不出来。
		如果要使用中易公司的字体,
		则建议使用 Windows Vista 及其以后版本提供的%
		宋体、黑体、楷体和仿宋体,
		以及 Microsoft Office 2003 及其以后版本提供的隶书和幼圆体,
		这些字体是 GB18030 字符集的,不存在上述问题。%
	}

	Linux 用户可以从软件源获得 GNU 的 make;
	其它类 UNIX 系统应该也会提供 make 工具,请参阅相应的文档以获得帮助。%
	Windows 用户可以从以下地址下载 Windows 下的 GNU make 工具:\\
	\hspace*{\parindent}%
	\url{http://gnuwin32.sourceforge.net/packages/make.htm}

	\section{安装方法}\label{sec:inst}

	使用 \TeX{}Live 的用户可以通过在终端中以管理员权限执行
\begin{Verbatim}[frame = single]
tlmgr install pkuthss
\end{Verbatim}
	来安装 pkuthss 文档模版。

	使用 \CTeX{} 套装的用户可以通过在命令提示符中执行
\begin{Verbatim}[frame = single]
mpm --install=pkuthss
\end{Verbatim}
	来安装 pkuthss 文档模版。

	\section{模版文件}\label{sec:doc-dir}

	在正确安装 pkuthss 文档模版之后,在终端/命令提示符中执行
\begin{Verbatim}[frame = single]
texdoc pkuthss
\end{Verbatim}
	所打开的 pdf 所在的同一目录中包含本文档(\verb|pkuthss.pdf|)的源代码%
	(\verb|utf8lf/| 和 \verb|gbkcrlf/| 两个子目录,
	两目录中代码除使用的编码和换行符外基本上完全相同)。

	其中,%
	\verb|utf8lf/| 目录下的源代码使用的是 UTF-8 编码、\verb|\n|(LF)换行,
	适合类 UNIX 系统用户使用;%
	\verb|gbkcrlf/| 目录下的源代码使用的是 GBK 编码、\verb|\r\n|(CRLF)换行,
	适合 Windows 用户使用。%
	用户可以试情况将 \verb|utf8lf/| 或 \verb|gbkcrlf/| 中
	的所有内容复制到合适的目录,
	并在此目录中根据模版修改出自己的论文。

	\verb|utf8lf/| 和 \verb|gbkcrlf/| 目录中的重要文件有:
	\begin{itemize}
		\item \verb|Makefile|:
			被 Make 工具调用的 Makefile,用于使编译工作自动化。
		\item \verb|Make.bat|:%
			Windows 下的伪“Makefile”,用 Windows 批处理实现。

		\item \verb|chap/|:目录,包含各章节内容:
		\begin{itemize}
			\item \verb|copyright.tex|:
				版权声明部分\footnote{%
					本文档中的版权声明并不是%
					学校默认要求的形式\supercite{pku-copyright}。
					符合学校要求的一个版权声明已经放在此文件中,
					但用 \texttt{\string\iffalse{} ...\ \string\fi} %
					注释掉了,
					用户可以考虑使用这个版本。
				}。
			\item \verb|originauth.tex|:
				原创性声明和使用授权说明部分\supercite{pku-originauth}。
		\end{itemize}
		\myemph{%
			注:%
			pkuthss 文档模板支持排版学校要求的二维码,
			请参考 \texttt{copyright.tex} 和
			\texttt{originauth.tex} 中的相关注释。%
		}

		\item \verb|img/|:目录,包含论文中所有图片:
		\begin{itemize}
			\item \verb|Makefile|:图片部分的 Makefile。
			\item \verb|Make.bat|:%
				Windows 下的伪“Makefile”,用 Windows 批处理实现。
			\item \verb|pkulogo.eps|:北京大学校徽。
			\item \verb|pkuword.eps|:“北京大学”字样。
		\end{itemize}
		\myemph{%
			注:%
			pdf\LaTeX{} 方式编译(见第 \ref{sec:compile} 节)
			可能需要将 eps 图片转换为 pdf 格式,
			而使用 Makefile 或伪“Makefile”时这些图片可以自动生成;
			不使用 Makefile 的用户可以手动运行 %
			\texttt{img/} 目录中的 Makefile 或伪“Makefile”
			来生成这些图片。%
		}
	\end{itemize}

	\section{编译方式}\label{sec:compile}

	pkuthss 文档模版支持三种编译方式,即
	\begin{itemize}
		\item \LaTeX{} -- dvipdfmx 方式:\\
			依次执行 \verb|latex|,\verb|biber|(或 \verb|bibtex|),%
			\verb|latex|,\verb|latex| 和 \verb|dvipdfmx|。
		\item pdf\LaTeX{} 方式:\\
			依次执行 \verb|pdflatex|,\verb|biber|(或 \verb|bibtex|),%
			\verb|pdflatex| 和 \verb|pdflatex|。
		\item Xe\LaTeX{} 方式:\\
			依次执行 \verb|xelatex|,\verb|biber|(或 \verb|bibtex|),%
			\verb|xelatex| 和 \verb|xelatex|。\\
			\myemph{%
				注意:Xe\LaTeX{} 对非 UTF-8 的编码支持不好,
				因此 Xe\LaTeX{} 方式的编译不支持 GBK 编码。
			}
	\end{itemize}

	pkuthss 文档模版附带的 Makefile 中已经对这三种编译方式进行了完整的配置。
	用户只需要在 Makefile 中通过设定变量 \verb|JOBNAME| 的值%
	指定被编译的主文件名,
	并通过设定变量 \verb|LATEX| 的值指定采用哪种编译方式,
	即可通过在主文件所在目录调用 Make 工具来实现自动编译:
	如果是在类 UNIX 环境下,则用户应该调用的命令名为 \verb|make|;
	而如果是在 Windows 环境下,
	则用户应该调用的命令名可能为 \verb|mingw32-make|。

	用户如果不想配置 Windows 下的 GNU Make,
	则也可以使用由 Windows 批处理实现的伪“Makefile”,
	通过在主文件所在目录调用 \verb|make|\footnote{%
		Windows 将批处理文件作为可执行文件,
		调用时可以不显式地指出扩展名。%
	} 或直接双击 \verb|Make.bat| 的图标运行之。

