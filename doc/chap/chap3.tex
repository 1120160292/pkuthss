% vim:ts=4:sw=4
%
% Copyright (c) 2008-2009 solvethis
% Copyright (c) 2010-2012 Casper Ti. Vector
% Public domain.

\chapter{问题及其解决}
	\section{可能存在的问题}
		\subsection{上游宏包可能引起的问题}

		hyperref 宏包和一些宏包可能发生冲突。
		关于如何避免这些冲突,
		可以参考 hyperref 宏包的 README 文件。
		此文件通常和执行 \verb|texdoc hyperref| %
		时打开的 pdf 文件位于同一目录中。
		低于 1.02c 版本的 ctex 宏包中对 hyperref 的设置有些不周,
		因此文档类中对其进行了一些手动的处理。
		考虑到新版本 ctex 宏包将逐渐被更多人采用,
		进行这些处理的代码将在以后被删除,
		而改成直接调用 ctex 宏包的 \verb|hyperref| 选项。

		当启用 natbib 宏包的 \verb|super| 选项时,
		其 \verb|\citenum| 命令生成的引用序号前会有一个额外的空格。
		因此建议使用第 \ref{ssec:misc} 小节中不带方括号的方式来%
		排版非上标的引用序号。

		\subsection{文档格式可能存在的问题}

		研究生手册和其电子版\cite{pku-thesisstyle}要求的论文封面并不一致。
		这里以电子版为准。

		\subsection{其它可能存在的问题}

		使用 GBK 编码和 pdf\LaTeX{} 编译方式时需要用户%
		运行 \verb|gbk2uni| 程序来转换 .out 文件,
		否则生成的 pdf 书签可能乱码。
		考虑到用户可能没有 \verb|gbk2uni| 程序,
		默认的 \verb|Makefile| 和 \verb|Make.bat| 中将相关代码注释掉了,
		用户在确定自己有 \verb|gbk2uni| 程序时可去掉相应的注释。

	\section{反馈意见和建议}

	关于 pkuthss 文档模版的意见和建议,
	请在北大未名 BBS 的 MathTools 版或 %
	Google Code 上 pkuthss 项目的 issue tracker%
	\footnote{\url{http://code.google.com/p/caspervector/issues/list}.}%
	上提出,
	或通过电子邮件\footnote%
	{\href{mailto:CasperVector@gmail.com}{\texttt{CasperVector@gmail.com}}.}%
	告知 Casper Ti. Vector。
	上述三种反馈方法中,建议用户尽量采用靠前的方法。

	在进行反馈时,请尽量确保已经仔细阅读本文档中的说明。
	如果是通过 BBS 或电子邮件进行反馈,
	请在标题中说明是关于 pkuthss 文档模版的反馈;
	如果是通过 Google Code 进行反馈,
	请给 issue 加上 \verb|Proj-Pkuthss| 标签。
	如果是错误报告,
	请说明所使用 pkuthss 模版的版本、
	自己使用的操作系统和 \TeX{} 系统的类型和版本;
	同时强烈建议附上一个出错的最小例子及其相应的编译日志(\verb|.log| 文件),
	在文件较长时请使用附件。

