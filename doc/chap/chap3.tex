% vim:ts=4:sw=4
%
% Copyright (c) 2008-2009 solvethis
% Copyright (c) 2010-2012 Casper Ti. Vector
% Public domain.

\chapter{问题及其解决}
	\section{文档中已经提到的常见问题(按重要性排序)}

	在默认设置(启用 \verb|colorlinks| 选项)下,
	黑白打印时文档中的部分彩色链接可能会变成浅灰色,
	解决方式见第 \ref{ssec:extra} 小节。

	中文字体字库不全(只包含 GB2312 字符集内字符)时,
	生成的 pdf 文档中可能缺少部分字符,
	解决方式见第 \ref{sec:req} 节。

	\verb|img/| 目录中 eps 图片未转换为 pdf 格式时,%
	pdf\LaTeX{} 方式编译可能出错,
	解决方式见第 \ref{sec:doc-dir} 节。

	使用过旧的 \TeX{} 系统和各宏包,
	或使用某些 Linux 发行版软件仓库所提供的 \TeX{}Live 时,
	可能引起一些问题,
	详见第 \ref{sec:req} 节。

	文档默认情况下是双面模式,章末可能产生空白页,详见第 \ref{ssec:options} 小节。

	通过一些设置,还可以满足例如被引用的文献按照引用顺序排序,
	而未引用的文献按照英文文献在前、中文文献在后排序这样的需求,
	见第 \ref{ssec:thirdparty} 小节。

	一些高级设置,
	如封面中部分内容长度超过预设空间容量时的设置,
	见第 \ref{sec:advanced} 节。

	\section{其它可能存在的问题}
		\subsection{上游宏包可能引起的问题}

		biblatex 宏包\supercite{biblatex}会自行设定 \verb|\bibname|,
		故会覆盖通过 \verb|\CTEXoptions| 设定的参考文献列表标题。
		使用 biblatex 的用户可以使用 \verb|\printbibliography| 的
		\verb|title| 选项来手动设定参考文献列表的标题,例如:
\begin{Verbatim}[frame = single]
\printbibliography[title = {文献}, ...] % “...”为其它选项。
\end{Verbatim}

		hyperref 宏包\supercite{hyperref}和一些宏包可能发生冲突。
		关于如何避免这些冲突,可以参考 hyperref 宏包的文档。
		此文件通常和执行 \verb|texdoc hyperref| %
		时打开的 pdf 文件位于同一目录中。

		\subsection{文档格式可能存在的问题}

		研究生手册和其电子版\supercite{pku-thesisstyle}要求的论文封面并不一致。
		这里以电子版为准。

		\subsection{其它一些问题}

		使用 GBK 编码和 pdf\LaTeX{} 编译方式时需要用户%
		运行 \verb|gbk2uni| 程序来转换 \verb|.out| 文件,
		否则生成的 pdf 书签可能乱码。
		考虑到用户可能没有 \verb|gbk2uni| 程序,且有用户使用 UTF-8 编码,
		默认的 \verb|Makefile| 和 \verb|Make.bat| 中将相关代码注释掉了,
		用户可以自行去掉相应的注释。

	\section{反馈意见和建议}

	关于 pkuthss 文档模版的意见和建议,
	请在北大未名 BBS 的 MathTools 版或 %
	Google Code 上 pkuthss 项目的 issue tracker%
	\footnote{\url{http://code.google.com/p/caspervector/issues/list}.}%
	上提出,
	或通过电子邮件\footnote%
	{\href{mailto:CasperVector@gmail.com}{\texttt{CasperVector@gmail.com}}.}%
	告知 Casper Ti. Vector。
	上述三种反馈方法中,建议用户尽量采用靠前的方法。

	在进行反馈时,请尽量确保已经仔细阅读本文档中的说明。
	如果是通过 BBS 或电子邮件进行反馈,
	请在标题中说明是关于 pkuthss 文档模版的反馈;
	如果是通过 Google Code 进行反馈,
	请给 issue 加上 \verb|Proj-Pkuthss| 标签。
	如果是错误报告,
	请说明所使用 pkuthss 模版的版本、
	自己使用的操作系统和 \TeX{} 系统的类型和版本;
	同时强烈建议附上一个出错的最小例子及其相应的编译日志(\verb|.log| 文件),
	在文件较长时请使用附件。

