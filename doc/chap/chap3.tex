% vim:ts=4:sw=4
%
% Copyright (c) 2008-2009 solvethis
% Copyright (c) 2010-2013 Casper Ti. Vector
% Public domain.

\chapter{问题及其解决}
	\section{文档中已经提到的常见问题(按重要性排序)}

	北大图书馆要求上传的论文中目录采用黑色字体;
	在默认设置(启用 \verb|colorlinks| 选项)下,
	黑白打印时文档中的部分彩色链接可能会变成浅灰色,
	解决方式见第 \ref{ssec:extra} 小节。

	中文字体字库不全(只包含 GB2312 字符集内字符)时,
	生成的 pdf 文档中可能缺少部分字符,
	解决方式见第 \ref{sec:req} 节。

	\verb|img/| 目录中 eps 图片未转换为 pdf 格式时,%
	pdf\LaTeX{} 方式编译可能出错,
	解决方式见第 \ref{sec:doc-dir} 节。

	使用过旧的 \TeX{} 系统和各宏包,
	或使用某些 Linux 发行版软件仓库所提供的 \TeX{}Live 时,
	可能引起一些问题,
	详见第 \ref{sec:req} 节。

	文档默认情况下是双面模式,章末可能产生空白页,详见第 \ref{ssec:options} 小节。

	通过一些设置,还可以满足例如被引用的文献按照引用顺序排序,
	而未引用的文献按照英文文献在前、中文文献在后排序这样的需求,
	见第 \ref{ssec:thirdparty} 小节。

	无法使用 biber 的用户可以使用 bibtex,具体设置方式请见第 \ref{sec:compile} 节。

	一些高级设置,
	如封面中部分内容长度超过预设空间容量时的设置,
	见第 \ref{sec:advanced} 节。

	\section{其它可能存在的问题}
		\subsection{上游宏包可能引起的问题}

		biblatex 宏包\supercite{biblatex}会自行设定 \verb|\bibname|,
		故会覆盖通过 \verb|\CTEXoptions| 设定的参考文献列表标题。
		使用 biblatex 的用户可以使用 \verb|\printbibliography| 的
		\verb|title| 选项来手动设定参考文献列表的标题,例如:
\begin{Verbatim}[frame = single]
\printbibliography[title = {文献}, ...] % “...”为其它选项。
\end{Verbatim}

		hyperref 宏包\supercite{hyperref}和一些宏包可能发生冲突。
		关于如何避免这些冲突,
		可以参考 hyperref 宏包的 README 文件中的“Package Compatibility”一节。
		此文件通常和执行 \verb|texdoc hyperref| %
		时打开的 pdf 文件位于同一目录中。

		使用 Xe\LaTeX{} 的用户可能遇到形如(其中 xxxxxxxx 为具体字体名)
\begin{Verbatim}[frame = single, fontsize = {\small}]
!!!!!!!!!!!!!!!!!!!!!!!!!!!!!!!!!!!!!!!!!!!!!!!!
!
! fontspec error: "font-not-found"
!
! The font "xxxxxxxx" cannot be found.
!
! See the fontspec documentation for further information.
!
! For immediate help type H <return>.
!...............................................
\end{Verbatim}
		的错误。
		这种错误一般是(主要是非 Windows 平台的)用户采用了自定义的(包括大小写
		不同于原文件的)字体文件名,
		并改动 ctex.cfg 等配置文件之后没有在调用 pkuthss 文档类时加入
		\verb|nofonts| 选项,
		又使用 Xe\LaTeX{} 编译造成的,使用
\begin{Verbatim}[frame = single]
\documentclass[nofonts, ...]{pkuthss} % “...”代表其它的选项。
\end{Verbatim}
		即可解决此问题。

		biber 运行时有一定概率出现形如(目录名可能稍有不同)
\begin{Verbatim}[frame = single, fontsize = {\small}]
data source .../par-xxxxxxxx/cache-xxxxxxxx/
	inc/lib/Biber/LaTeX/recode_data.xml not found in .
\end{Verbatim}
		的错误。
		这种错误一般是 biber 在自解压阶段被终止之后,
		未删除临时目录 \verb|.../par-xxxxxxxx/| 就重新运行 biber 时出现。
		遇到这种情况时,删除掉上述临时目录及其所有内容,
		再重新运行 biber 通常便可解决问题。

		caption 宏包\supercite{caption}对于其不认识的宏包均会提示
\begin{Verbatim}[frame = single, fontsize = {\small}]
Package caption Warning: Unsupported document class (or package) detected,
(caption)                usage of the caption package is not recommended.
See the caption package documentation for explanation.
\end{Verbatim}
		pkuthss 文档模板基于 ctexbook 文档类,
		而后者基于标准的 book 文档类。
		因此,这个警告并不影响用户正常使用\footnote{%
			\url{http://bbs.ctex.org/forum.php?mod=redirect&goto=findpost&ptid=63117&pid=402145}.%
		}。

		\subsection{文档格式可能存在的问题}

		研究生手册和其电子版\supercite{pku-thesisstyle}要求的论文封面可能并不一致,
		这里以电子版为准。

		\subsection{其它一些问题}

		使用 GBK 编码和 pdf\LaTeX{} 编译方式时需要用户%
		运行 \verb|gbk2uni| 程序来转换 \verb|.out| 文件,
		否则生成的 pdf 书签可能乱码。
		考虑到用户可能没有 \verb|gbk2uni| 程序,且有用户使用 UTF-8 编码,
		默认的 \verb|Makefile| 和 \verb|Make.bat| 中将相关代码注释掉了,
		用户可以自行去掉相应的注释。

	\section{反馈意见和建议}

	关于 pkuthss 文档模版的意见和建议,
	请在北大未名 BBS 的 MathTools 版或 %
	Google Code 上 pkuthss 项目的 issue tracker%
	\footnote{\url{http://code.google.com/p/caspervector/issues/list}.}%
	上提出,
	或通过电子邮件\footnote%
	{\href{mailto:CasperVector@gmail.com}{\texttt{CasperVector@gmail.com}}.}%
	告知 Casper Ti. Vector。
	上述三种反馈方法中,建议用户尽量采用靠前的方法。

	在进行反馈时,请尽量确保已经仔细阅读本文档中的说明。
	如果是通过 BBS 或电子邮件进行反馈,
	请在标题中说明是关于 pkuthss 文档模版的反馈;
	如果是通过 Google Code 进行反馈,
	请给 issue 加上 \verb|Proj-Pkuthss| 标签。
	如果是错误报告,
	请说明所使用 pkuthss 模版的版本、
	自己使用的操作系统和 \TeX{} 系统的类型和版本;
	同时强烈建议附上一个出错的最小例子及其相应的编译日志(\verb|.log| 文件),
	在文件较长时请使用附件。

