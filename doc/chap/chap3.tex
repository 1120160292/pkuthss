\chapter{问题及其解决}
	\section{FAQ\label{sec:faq}}
	\begin{enumerate}
		\item[\textbf{Q:}]
		我的编译结果很奇怪,文字很靠近页面的顶端。请问这是怎么回事?

		\item[\textbf{A:}]
		请检查你的程序设置。
		如果使用 WinEdt,可点击 Options,选择 Execution Modes,
		检查一下 dvips、dvipdfmx、ps2pdf 等程序的纸张设置。

		\vspace{1em}
		\item[\textbf{Q:}]
		打印论文时不希望使用彩色的链接,请问应该怎么办?

		\item[\textbf{A:}]
		\verb|\hypersetup{colorlinks=false}|。
		关于书签和链接的问题,
		请参阅 hyperref 宏包的文档\cite{hyperref-doc}。

		\vspace{1em}
		\item[\textbf{Q:}]
		导言区的内容好多,应该有好多在我的论文里是不必要的。
		请问可以去掉哪些?

		\item[\textbf{A:}]
		如果你使用 GBK 编码,则 pkuthss 文档类的 UTF-8 选项是不必要的。
		如果你不需要生成的 pdf 里的书签和链接,则 hyperref 宏包是不必要的,
		同时用于进行相关设置的 %
		\verb|\hypersetup| 和 \verb|\pdfbookmark| 命令也应该去掉。
		如果你不使用 \verb|\verbatiminput| 命令和 \verb|comment| 环境,
		则 verbatim 宏包是不需要的。
		如果你不需要上标的引用记号,则 \verb|\cite| 宏可以去掉。
		如果你不需要使用密集的罗列环境,则 \verb|| 宏可以去掉。
		如果你不需要多次在文档中引用其版本,
		则 \verb|\docversion| 宏可以去掉。

		wasysym 宏包不应该去掉,
		因为 \verb|chap/originauth.tex| 中使用了其提供的 \verb|\Box| 命令。
		设置页面居中和行距的命令非常不建议去掉:
		如果改变这些设置,虽然不会对排版效果造成致命的影响,
		但影响可能还是很显著的。

		\vspace{1em}
		\item[\textbf{Q:}]
		文档里面“致谢”一章的书签链接到的位置不对,请问这是为什么?

		\item[\textbf{A:}]
		这是由上游的 ctex 宏包的一个问题造成的:
		在 \verb|\backmatter| 以后,
		即使用 \verb|\chapter| 命令开始的章节也不会被编号,
		但会计入目录和产生书签。
		使用低于 1.02 版的 ctexbook 文档类时,
		产生的 pdf 文档的这一类书签和链接指向的位置常常是错误的。
		如果你不能将 ctex 宏包更新到 1.02 及以后的版本,
		一个缓解问题的办法是将 \verb|\backmatter| 以后的 %
		\verb|\chapter| 命令全部改为 \verb|\specialchap| 命令。
	\end{enumerate}

	\section{可能存在的问题}

	研究生手册和其电子版\cite{pku-thesisstyle}要求的论文封面并不一致。
	这里以电子版为准。

	\section{反馈意见和建议}

	关于 pkuthss 文档模板的意见和建议,
	请到北大未名 BBS 的 MathTools 版提出。
	谢谢 :)

