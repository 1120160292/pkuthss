\chapter{pkuthss 文档模版的实现}
\raggedbottom % 避免某些奇怪的“Underfull \vbox”警告。

	\section{pkuthss 文档类的实现}
	源文件被列出的顺序是按照其被调用的顺序,
	代码前面 26 行共用的文件头(主要为版权声明)均没有列出。

		\subsection{\texttt{pkuthss.cls}}
		\VerbatimCode{pkuthss.cls}

		\subsection{\texttt{pkuthss-option.def}}
		\VerbatimCode{pkuthss-option.def}

		\subsection{\texttt{pkuthss-common.def}}
		\VerbatimCode{pkuthss-common.def}

		\subsection{\texttt{pkuthss-utf8.def} 和 \texttt{pkuthss-gbk.def}}
		\VerbatimCode{pkuthss-utf8.def}

		\subsection{\texttt{pkuthss-extra.sty}}
		\VerbatimCode{pkuthss-extra.sty}

	\section{其余部分} % FIXME

	pkuthss 文档模板的源代码中,除已经有了较为详细的注释,
	故请直接参照相应文件中的注释。

	\emph
	{%
		注:%
		\texttt{img/} 目录中的 %
		\texttt{Makefile} 和两个 PostScript(\texttt{.ps})文件中%
		也有详细的注释哦 :)
	}

\flushbottom % 取消 \raggedend 的作用。

