% vim:ts=4:sw=4
%
% Copyright (c) 2008-2009 solvethis
% Copyright (c) 2010-2012 Casper Ti. Vector
% Public domain.

\chapter{pkuthss 文档模版的实现}
\raggedbottom % 避免某些奇怪的“Underfull \vbox”警告。

	\section{pkuthss 文档类和 pkuthss-extra 宏包的实现}	
		\subsection{共用文件头部}
		\VerbatimInput[
			frame = lines, fontsize = {\footnotesize}, tabsize = 2,
			baselinestretch = 1.25, lastline = 23, numbers = left
		]{pkuthss.cls}

		\subsection{\texttt{pkuthss.cls}}
		\VerbatimInput[
			frame = lines, fontsize = {\footnotesize}, tabsize = 2,
			baselinestretch = 1.25, firstline = 25, numbers = left
		]{pkuthss.cls}

		\subsection{\texttt{pkuthss-utf8.def} 和 \texttt{pkuthss-gbk.def}}
		\VerbatimInput[
			frame = lines, fontsize = {\footnotesize}, tabsize = 2,
			baselinestretch = 1.25, firstline = 25, numbers = left
		]{pkuthss-utf8.def}

		\subsection{\texttt{pkuthss-extra.sty}}
		\VerbatimInput[
			frame = lines, fontsize = {\footnotesize}, tabsize = 2,
			baselinestretch = 1.25, firstline = 25, numbers = left
		]{pkuthss-extra.sty}

	\section{pkuthss 说明(示例)文档的源代码}

	本文档的源代码中大部分已经有了较为详细的注释,
	故请直接参照相应文件中的注释。

	\emph
	{%
		注:%
		\texttt{img/} 目录中的 \texttt{Makefile} 和%
		两个 PostScript(\texttt{.eps})文件(都是文本文件)中%
		也有详细的注释哦 :)
	}

\flushbottom % 取消 \raggedbottom 的作用。

