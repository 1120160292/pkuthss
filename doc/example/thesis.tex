% Copyright (c) 2008-2009 solvethis
% Copyright (c) 2010-2016,2018 Casper Ti. Vector
% Public domain.
%
% 使用前请先仔细阅读 pkuthss 和 biblatex-caspervector 的文档,
% 特别是其中的 FAQ 部分和用红色强调的部分。
% 两者可在终端/命令提示符中用
%   texdoc pkuthss
%   texdoc biblatex-caspervector
% 调出。

% 采用了自定义的(包括大小写不同于原文件的)字体文件名,
% 并改动 ctex.cfg 等配置文件的用户请自行加入 nofonts 选项;
% 其它用户不用加入 nofonts 选项,加入之后反而会产生错误。
\documentclass[UTF8]{pkuthss}

% 使用 biblatex 排版参考文献,并规定其格式(详见 biblatex-caspervector 的文档)。
% 这里按照西文文献在前,中文文献在后排序(“sorting = ecnyt”);
% 若需按照中文文献在前,西文文献在后排序,请设置“sorting = cenyt”;
% 若需按照引用顺序排序,请设置“sorting = none”。
% 若需在排序中实现更复杂的需求,请参考 biblatex-caspervector 的文档。
\usepackage[backend = biber, style = caspervector, utf8, sorting = ecnyt]{biblatex}

% 按学校要求设定参考文献列表中的条目之内及之间的距离。
\setlength{\bibitemsep}{3bp}
% 对于 linespread 值的计算过程有兴趣的同学可以参考 pkuthss.cls。
\renewcommand*{\bibfont}{\zihao{5}\linespread{1.27}\selectfont}

% 设定文档的基本信息。
\pkuthssinfo{
	cthesisname = {博士研究生学位论文}, ethesisname = {Doctor Thesis},
	ctitle = {测试文档}, etitle = {Test Document},
	cauthor = {某某},
	eauthor = {Test},
	studentid = {0123456789},
	date = {某年某月},
	school = {某某学院},
	cmajor = {某某专业}, emajor = {Some Major},
	direction = {某某方向},
	cmentor = {某某教授}, ementor = {Prof.\ Somebody},
	ckeywords = {其一,其二}, ekeywords = {First, Second}
}
% 载入参考文献数据库(注意不要省略“.bib”)。
\addbibresource{thesis.bib}

% 普通用户可删除此段,并相应地删除 chap/*.tex 中的
% “\pkuthssffaq % 中文测试文字。”一行。
\usepackage{color}
\def\pkuthssffaq{%
	\emph{\textcolor{red}{pkuthss 文档模版最常见问题:}}

	\texttt{\string\cite}、\texttt{\string\parencite} %
	和 \texttt{\string\supercite} 三个命令分别产生%
	未格式化的、带方括号的和上标且带方括号的引用标记:%
	\cite{test-en},\parencite{test-zh}、\supercite{test-en, test-zh}。

	若要避免章末空白页,请在调用 pkuthss 文档类时加入 \texttt{openany} 选项。

	如果编译时不出参考文献,
	请参考 \texttt{texdoc pkuthss}“问题及其解决”一章
	“上游宏包可能引起的问题”一节中关于 biber 的说明。%
}

\begin{document}
	% 以下为正文之前的部分,默认不进行章节编号。
	\frontmatter
	% 此后到下一 \pagestyle 命令之前不排版页眉或页脚。
	\pagestyle{empty}
	% 自动生成封面。
	\maketitle
	% 版权声明。封面要求单面打印,故需新开右页。
	\cleardoublepage
	% Copyright (c) 2008-2009 solvethis
% Copyright (c) 2010-2017 Casper Ti. Vector
% All rights reserved.
%
% Redistribution and use in source and binary forms, with or without
% modification, are permitted provided that the following conditions are
% met:
%
% * Redistributions of source code must retain the above copyright notice,
%   this list of conditions and the following disclaimer.
% * Redistributions in binary form must reproduce the above copyright
%   notice, this list of conditions and the following disclaimer in the
%   documentation and/or other materials provided with the distribution.
% * Neither the name of Peking University nor the names of its contributors
%   may be used to endorse or promote products derived from this software
%   without specific prior written permission.
%
% THIS SOFTWARE IS PROVIDED BY THE COPYRIGHT HOLDERS AND CONTRIBUTORS "AS
% IS" AND ANY EXPRESS OR IMPLIED WARRANTIES, INCLUDING, BUT NOT LIMITED TO,
% THE IMPLIED WARRANTIES OF MERCHANTABILITY AND FITNESS FOR A PARTICULAR
% PURPOSE ARE DISCLAIMED. IN NO EVENT SHALL THE COPYRIGHT HOLDER OR
% CONTRIBUTORS BE LIABLE FOR ANY DIRECT, INDIRECT, INCIDENTAL, SPECIAL,
% EXEMPLARY, OR CONSEQUENTIAL DAMAGES (INCLUDING, BUT NOT LIMITED TO,
% PROCUREMENT OF SUBSTITUTE GOODS OR SERVICES; LOSS OF USE, DATA, OR
% PROFITS; OR BUSINESS INTERRUPTION) HOWEVER CAUSED AND ON ANY THEORY OF
% LIABILITY, WHETHER IN CONTRACT, STRICT LIABILITY, OR TORT (INCLUDING
% NEGLIGENCE OR OTHERWISE) ARISING IN ANY WAY OUT OF THE USE OF THIS
% SOFTWARE, EVEN IF ADVISED OF THE POSSIBILITY OF SUCH DAMAGE.

% 此处不用 \specialchap,因为学校要求目录不包括其自己及其之前的内容。
\chapter*{版权声明}
% 综合学校的书面要求及 Word 模版来看,版权声明页不需加页眉、页脚。
\thispagestyle{empty}

任何收存和保管本论文各种版本的单位和个人,
未经本论文作者同意,不得将本论文转借他人,
亦不得随意复制、抄录、拍照或以任何方式传播。
否则一旦引起有碍作者著作权之问题,将可能承担法律责任。

% 若需排版二维码,请将二维码图片重命名为“barcode”,
% 转为合适的图片格式,并放在当前目录下,然后去掉下面 2 行的注释。
%\vfill\noindent
%\includegraphics[height = 5em]{barcode}

% vim:ts=4:sw=4


	% 此后到下一 \pagestyle 命令之前正常排版页眉和页脚。
	\cleardoublepage
	\pagestyle{plain}
	% 重置页码计数器,用大写罗马数字排版此部分页码。
	\setcounter{page}{0}
	\pagenumbering{Roman}
	% 中西文摘要。
	% Copyright (c) 2014,2016 Casper Ti. Vector
% Public domain.

\begin{cabstract}
	\pkuthssffaq % 中文测试文字
\end{cabstract}

\begin{eabstract}
	Test of the English abstract.
\end{eabstract}

% vim:ts=4:sw=4

	% 自动生成目录。
	\tableofcontents

	% 以下为正文部分,默认要进行章节编号。
	\mainmatter
	% 序言。
	% Copyright (c) 2014,2016 Casper Ti. Vector
% Public domain.

\specialchap{序言}
\pkuthssffaq % 中文测试文字。

% vim:ts=4:sw=4

	% 各章节。
	\chapter{使用介绍}
	\section{重要文件}

	本文档所在目录下各重要文件如下:
	\begin{itemize}\denselist
		\item \verb|pkuthss.cls|:pkuthss~文档类的类文件。
		\item \verb|pkuthss-gbk.def|:
			在~\verb|pkuthss.cls|~中使用的定义文件,用于~GBK~编码。
		\item \verb|pkuthss-utf8.def|:
			在~\verb|pkuthss.cls|~中使用的定义文件,用于~UTF-8~编码。
		\item \verb|sample.tex|:主文件,编译该文件即可。
		\item \verb|sample.pdf|:即本文档,由编译~\verb|sample.tex|~得到。
		\item \verb|Makefile|:Makefile,用于使编译工作自动化。
		\item \verb|Make.bat|:%
			Windows~下的伪“Makefile”,由~Windows~的批处理实现。
		\item \verb|chap/|:文件夹,包含各章节内容:
		\begin{itemize}\denselist
			\item \verb|copyright.tex|:版权声明部分\footnote%
			{\ %
				因为本文档的许可证限制,我们必须附上许可证的文本;
				但用户可能选择其它类型的版权声明,
				故~\texttt{license/}\linebreak[1]~目录不是必需的。
				一个可能更常用的版权声明已经放在此文件中,但被注释掉了,
				用户可以考虑使用那个版本。
				如果使用那个版本,就不再需要~\texttt{license/}~目录了。
			}。
			\item \verb|originauth.tex|:
				原创性声明和使用授权说明部分~\supercite{F11}。
		\end{itemize}
		\item \verb|img/|:文件夹,包含论文中所有图片:
		\begin{itemize}\denselist
			\item \verb|Makefile|:图片部分的~Makefile。
			\item \verb|Make.bat|:%
				Windows~下的伪“Makefile”,由~Windows~的批处理实现。
			\item \verb|pkulogo.ps|:北大校徽。
			\item \verb|pkuword.ps|:“北京大学”字样。
		\end{itemize}
	\end{itemize}

	\section{系统要求}

	正确编译需要以下几部分:
	\begin{itemize}\denselist
		\item 一个基本的~\LaTeX{}~发行版。
		\item CJK~或~xeCJK(供~Xe\LaTeX{}~使用)宏包。
		\item ctex~宏包\supercite{ctex-doc,ctexfaq}%
			(提供了~ctexbook~文档类)。
		\item 中文字体。
		\item 如果需要使用~Makefile~来实现自动编译,还需要~Make~工具;
			但如果使用由批处理实现的伪“Makefile”就不用了。
	\end{itemize}

	最新的~\TeX{}Live~系统和~\CTeX~套装都已经包含%
	除中文字体之外所有要求的项目;中文字体需要用户自行获得。

	Linux~用户可以从软件源获得~GNU~的~make;
	其它类~UNIX~系统应该也会提供~make~工具,请参阅相应的文档以获得帮助。%
	Windows~用户可以从以下地址下载~Windows~下的~GNU make~工具:

	\url{http://gnuwin32.sourceforge.net/packages/make.htm}(国际网)
	\vspace{-0.1em}\par
	\url{http://c.pku.edu.cn/software/c/mingw-c.7z}\footnote%
	{\ 感谢曹东刚老师在教学网站提供~GNU make~的下载。}(北大校园网)

	为了获得最好的支持,我们建议用户使用最新版的~\LaTeX{}~系统和各宏包%
	\footnote%
	{\ %
		使用~\TeX{}Live 2009~及其以上稳定版本的用户可以正常使用本模板;
		从~\TeX{}Live~的稳定(即不是“pretest”)源更新%
		到~CTAN~上最新稳定版本的用户也可以正常使用本模板。
	}。

	\section{编译方式}

	pkuthss~文档模板支持三种编译方式,即
	\begin{itemize}\denselist
	  \item \LaTeX{} -- dvipdf~方式:
		依次执行~\verb|latex|,\verb|bibtex|,%
		\verb|latex|,\verb|latex|,\verb|dvipdfmx|。
	  \item pdf\LaTeX{}~方式:
		依次执行~\verb|pdflatex|,\verb|bibtex|,%
		\verb|pdflatex|,\verb|pdflatex|。
	  \item Xe\LaTeX{}~方式:
		依次执行~\verb|xelatex|,\verb|bibtex|,%
		\verb|xelatex|,\verb|xelatex|。%
		\emph
		{%
			注意:Xe\LaTeX{}~对非~UTF-8~的编码支持不好,
			因此Xe\LaTeX{}~方式的编译不支持~GBK~编码。
		}
	\end{itemize}

	pkuthss~文档模板附带的~Makefile~中已经对这三种编译方式进行了完整的配置。
	用户只需要在~Makefile~中通过设定变量~\verb|JOBNAME|~的值%
	指定被编译的主文件名,
	并通过设定变量~\verb|LATEX|~的值指定采用哪种编译方式,
	即可通过在主文件所在目录调用~Make~工具来实现自动编译:
	如果是在类~UNIX~环境下,则用户应该调用的命令名为~\verb|make|;
	而如果是在~Windows~环境下,
	则用户应该调用的命令名为~\verb|mingw32-make|。

	用户如果不想配置~Windows~下的~GNU Make,
	则也可以使用由~Windows~批处理实现的伪“Makefile”,
	通过在主文件所在目录调用~\verb|make|\footnote%
	{\ %
		Windows~将批处理文件作为可执行文件,
		调用时可以不显式地指出扩展名。
	}~或直接双击~\verb|make.bat|~的图标运行之。%
	\emph
	{%
		注意:这样不能自动生成编译所需的部分图片。
		用户可能需要进入~\texttt{img/}~目录%
		执行那里的~\texttt{make.bat}~来手动生成这些图片。
	}


	% 结论。
	% Copyright (c) 2014,2016 Casper Ti. Vector
% Public domain.

\specialchap{结论}
\pkuthssffaq % 中文测试文字。

% vim:ts=4:sw=4


	% 正文中的附录部分。
	\appendix
	% 排版参考文献列表。bibintoc 选项使“参考文献”出现在目录中;
	% 如果同时要使参考文献列表参与章节编号,可将“bibintoc”改为“bibnumbered”。
	\printbibliography[heading = bibintoc]
	% 各附录。
	\chapter{更新记录}

\verbatiminput{ChangeLog.txt}



	% 以下为正文之后的部分,默认不进行章节编号。
	\backmatter
	% 致谢。
	% Copyright (c) 2014,2016 Casper Ti. Vector
% Public domain.

\chapter{致谢}
\pkuthssffaq % 中文测试文字。

% vim:ts=4:sw=4

	% 原创性声明和使用授权说明。
	../example/chap/origin.tex
\end{document}

% vim:ts=4:sw=4
