% vim:ts=4:sw=4
%
% Copyright (c) 2008-2009 solvethis
% Copyright (c) 2010-2014 Casper Ti. Vector
% Public domain.
%
% 使用前请先仔细阅读 pkuthss 和 biblatex-caspervector 的文档,
% 特别是其中的 FAQ 部分和用红色强调的部分。
% 两者可在终端/命令提示符中用
%   texdoc pkuthss
%   texdoc biblatex-caspervector
% 调出。

% 采用了自定义的(包括大小写不同于原文件的)字体文件名,
% 并改动 ctex.cfg 等配置文件的用户请自行加入 nofonts 选项;
% 其它用户不用加入 nofonts 选项,加入之后反而会产生错误。
%
% 图书馆要求电子版论文的目录必须为黑色,
% 且某些教务要求打印版论文的文字部分为纯黑色而非灰度打印,
% 【因此最终打印和提交论文前,请将“colorlinks”改为“nocolorlinks”。】
\documentclass[UTF8, colorlinks]{pkuthss}

% 使用 biblatex 排版参考文献,并规定其格式。
%
% 如果无法使用 biber,可以把“backend = biber”改为“backend = bibtex”,
% 并改用 bibtex 产生参考文献,详见 pkuthss 的文档。
% 使用 biber 时,请去掉所有的 sorting 选项,否则会出错。
%
% 默认按照引用顺序排序(“sorting = none”),详见 biblatex-caspervector 的文档
% (因为是默认设置所以其实不用写,不过出于完备性的考虑仍然在这里列出)。
% 若需要按照英文文献在前,中文文献在后排序,请设置“sorting = ecnty”;
% 若需要按照中文文献在前,英文文献在后排序,请设置“sorting = centy”。
\usepackage[backend = biber, style = caspervector, utf8, sorting = none]{biblatex}
% 产生 originauth.tex 里的 \square。
\usepackage{amssymb}

% 设定文档的基本信息。
\pkuthssinfo{
	cthesisname = {本科生毕业论文}, ethesisname = {Undergraduate Thesis},
	ctitle = {测试文档}, etitle = {Test Document},
	cauthor = {某某},
	eauthor = {Test},
	studentid = {0123456789},
	date = {某年某月},
	school = {某某学院},
	cmajor = {某某专业}, emajor = {Some Major},
	direction = {某某方向},
	cmentor = {某某教授}, ementor = {Prof.\ Somebody},
	ckeywords = {某某}, ekeywords = {Something}
}
% 导入参考文献数据库(注意不要省略“.bib”)。
\addbibresource{thesis.bib}

% 普通用户可删除此段。
\usepackage{color}
\def\pkuthssffaq{%
	\emph{\textcolor{red}{pkuthss 文档模版最常见问题:}}

	最终打印和提交论文前,
	请将 pkuthss 文档类选项中的 %
	\texttt{colorlinks} 改为 \texttt{nocolorlinks},
	因为图书馆要求电子版论文的目录必须为黑色,
	且某些教务要求打印版论文的文字部分为纯黑色而非灰度打印。

	\texttt{\string\cite}、\texttt{\string\parencite} %
	和 \texttt{\string\supercite} 三个命令分别产生%
	未格式化的、带方括号的和上标且带方括号的引用标记:%
	\cite{test-en},\parencite{test-zh}、\supercite{test-en, test-zh}。%

	若要避免章末空白页,请在调用 pkuthss 文档类时加入 \texttt{oneside} 选项。
}

\begin{document}
	% 以下为正文之前的部分。
	\frontmatter

	% 自动生成标题页。
	\maketitle
	% 版权声明。
	\cleardoublepage
\chapter*{\textbf{版权声明}}
{
	\zihao{3}
\begin{comment}
	任何收存和保管本论文各种版本的单位和个人,
	未经本论文作者同意,不得将本论文转借他人,
	亦不得随意复制、抄录、拍照或以任何方式传播。
	否则一旦引起有碍作者著作权之问题,将可能承担法律责任。
\end{comment}
	版权所有~\copyright~2008--2009 solvethis
	\par
	版权所有~\copyright~2010 Casper Ti. Vector
	\vspace{0.75em}\par
	本文档可在~GNU~自由文档许可证(GFDL)\footnote%
	{\ \url{http://www.fsf.org/licensing/licenses/fdl.html}}%
	的第~1.3~版(或之后任意版本)或~GNU~通用公共许可证(GPL)\footnote%
	{\ \url{http://www.fsf.org/licensing/licenses/gpl.html}}%
	的第~3~版(或之后任意版本)所规定的条款下自由地复制、修改和发布。
	
	以上所述两个许可证应该在本文档所在目录的~\verb|license/|~%
	\linebreak[1]子目录下,
	文件名分别为~\verb|fdl-1.3.txt|~和~\verb|gpl-3.0.txt|。
	如果没有,你可以到上面提到的网址查看许可证内容。
	如果还不行,请写信给下面的地址以获得邮寄的许可证:
\begin{verbatim}
    The Free Software Foundation, Inc.,
    675 Mass Ave, Cambridge, MA02139, USA 
\end{verbatim}
	\par
}


	% 中英文摘要。
	\begin{abstract}

%%内容摘要要求在3000字以内, 应简要说明本论文的目的、内容、方法、成果和结论.
%%要突出论文的创新之处.

本文简要介绍了pkuthss.cls这个文档类的情况, 并
以实例说明了它的使用方法.

\end{abstract}

	% 自动生成目录。
	\tableofcontents

	% 以下为正文。
	\mainmatter

	% 序言。
	\specialchap{绪言}

本文档是“北京大学论文文档模板”的测试和说明文档。

在此之前,包括~dypang\supercite{dypang}、FerretL\supercite{FerretL}、
lwolf\supercite{lwolf}、Langpku\supercite{Langpku}、
solvethis\supercite{solvethis}~等的数位网友均做过学位论文模板工作。
本论文模板是~solvethis~的~pkuthss~模板的更新版本,
更新的重点是重构和对新文档类、宏包的支持。

pkuthss~文档模板现在的维护者是~Casper Ti. Vector\footnote%
{\ \href{CasperVector@gmail.com}{\texttt{CasperVector@gmail.com}}}。


	% 各章节。
	\chapter{使用介绍}
	\section{重要文件}

	本文档所在目录下各重要文件如下:
	\begin{itemize}\denselist
		\item \verb|pkuthss.cls|:pkuthss~文档类的类文件。
		\item \verb|pkuthss-gbk.def|:
			在~\verb|pkuthss.cls|~中使用的定义文件,用于~GBK~编码。
		\item \verb|pkuthss-utf8.def|:
			在~\verb|pkuthss.cls|~中使用的定义文件,用于~UTF-8~编码。
		\item \verb|sample.tex|:主文件,编译该文件即可。
		\item \verb|sample.pdf|:即本文档,由编译~\verb|sample.tex|~得到。
		\item \verb|Makefile|:Makefile,用于使编译工作自动化。
		\item \verb|Make.bat|:%
			Windows~下的伪“Makefile”,由~Windows~的批处理实现。
		\item \verb|chap/|:文件夹,包含各章节内容:
		\begin{itemize}\denselist
			\item \verb|copyright.tex|:版权声明部分\footnote%
			{\ %
				因为本文档的许可证限制,我们必须附上许可证的文本;
				但用户可能选择其它类型的版权声明,
				故~\texttt{license/}\linebreak[1]~目录不是必需的。
				一个可能更常用的版权声明已经放在此文件中,但被注释掉了,
				用户可以考虑使用那个版本。
				如果使用那个版本,就不再需要~\texttt{license/}~目录了。
			}。
			\item \verb|originauth.tex|:
				原创性声明和使用授权说明部分~\supercite{F11}。
		\end{itemize}
		\item \verb|img/|:文件夹,包含论文中所有图片:
		\begin{itemize}\denselist
			\item \verb|Makefile|:图片部分的~Makefile。
			\item \verb|Make.bat|:%
				Windows~下的伪“Makefile”,由~Windows~的批处理实现。
			\item \verb|pkulogo.ps|:北大校徽。
			\item \verb|pkuword.ps|:“北京大学”字样。
		\end{itemize}
	\end{itemize}

	\section{系统要求}

	正确编译需要以下几部分:
	\begin{itemize}\denselist
		\item 一个基本的~\LaTeX{}~发行版。
		\item CJK~或~xeCJK(供~Xe\LaTeX{}~使用)宏包。
		\item ctex~宏包\supercite{ctex-doc,ctexfaq}%
			(提供了~ctexbook~文档类)。
		\item 中文字体。
		\item 如果需要使用~Makefile~来实现自动编译,还需要~Make~工具;
			但如果使用由批处理实现的伪“Makefile”就不用了。
	\end{itemize}

	最新的~\TeX{}Live~系统和~\CTeX~套装都已经包含%
	除中文字体之外所有要求的项目;中文字体需要用户自行获得。

	Linux~用户可以从软件源获得~GNU~的~make;
	其它类~UNIX~系统应该也会提供~make~工具,请参阅相应的文档以获得帮助。%
	Windows~用户可以从以下地址下载~Windows~下的~GNU make~工具:

	\url{http://gnuwin32.sourceforge.net/packages/make.htm}(国际网)
	\vspace{-0.1em}\par
	\url{http://c.pku.edu.cn/software/c/mingw-c.7z}\footnote%
	{\ 感谢曹东刚老师在教学网站提供~GNU make~的下载。}(北大校园网)

	为了获得最好的支持,我们建议用户使用最新版的~\LaTeX{}~系统和各宏包%
	\footnote%
	{\ %
		使用~\TeX{}Live 2009~及其以上稳定版本的用户可以正常使用本模板;
		从~\TeX{}Live~的稳定(即不是“pretest”)源更新%
		到~CTAN~上最新稳定版本的用户也可以正常使用本模板。
	}。

	\section{编译方式}

	pkuthss~文档模板支持三种编译方式,即
	\begin{itemize}\denselist
	  \item \LaTeX{} -- dvipdf~方式:
		依次执行~\verb|latex|,\verb|bibtex|,%
		\verb|latex|,\verb|latex|,\verb|dvipdfmx|。
	  \item pdf\LaTeX{}~方式:
		依次执行~\verb|pdflatex|,\verb|bibtex|,%
		\verb|pdflatex|,\verb|pdflatex|。
	  \item Xe\LaTeX{}~方式:
		依次执行~\verb|xelatex|,\verb|bibtex|,%
		\verb|xelatex|,\verb|xelatex|。%
		\emph
		{%
			注意:Xe\LaTeX{}~对非~UTF-8~的编码支持不好,
			因此Xe\LaTeX{}~方式的编译不支持~GBK~编码。
		}
	\end{itemize}

	pkuthss~文档模板附带的~Makefile~中已经对这三种编译方式进行了完整的配置。
	用户只需要在~Makefile~中通过设定变量~\verb|JOBNAME|~的值%
	指定被编译的主文件名,
	并通过设定变量~\verb|LATEX|~的值指定采用哪种编译方式,
	即可通过在主文件所在目录调用~Make~工具来实现自动编译:
	如果是在类~UNIX~环境下,则用户应该调用的命令名为~\verb|make|;
	而如果是在~Windows~环境下,
	则用户应该调用的命令名为~\verb|mingw32-make|。

	用户如果不想配置~Windows~下的~GNU Make,
	则也可以使用由~Windows~批处理实现的伪“Makefile”,
	通过在主文件所在目录调用~\verb|make|\footnote%
	{\ %
		Windows~将批处理文件作为可执行文件,
		调用时可以不显式地指出扩展名。
	}~或直接双击~\verb|make.bat|~的图标运行之。%
	\emph
	{%
		注意:这样不能自动生成编译所需的部分图片。
		用户可能需要进入~\texttt{img/}~目录%
		执行那里的~\texttt{make.bat}~来手动生成这些图片。
	}


	% 结论。
	\specialchap{结论}

pkuthss~文档模板结构较为简洁、清晰、灵活,较为易于学习和使用。
希望它能为各位需要使用~\LaTeX{}~撰写学位论文的同学带来帮助。



	% 正文中的附录部分。
	\appendix
	% 排版参考文献列表。
	\printbibliography[
		% 使“参考文献”出现在目录中;如果同时要使参考文献列表参与章节编号,
		% 可将“bibintoc”改为“bibnumbered”。
		heading = bibintoc,
		% 单独设定排序方案。此设定会局部覆盖之前的全局设置。
		% 注:只有同时使用 2.x 或之后版本的 biblatex 和相应兼容版本的 biber,
		% 才能对每个 \printbibliography 命令采用不同的排序方案,
		% 否则只能在导入 biblatex 宏包时就(全局)指定排序方案。
		% 在这样的情况下,请去掉所有的 sorting 选项,否则可能出错。
		sorting = ecnty
	]
	% 各附录。
	\chapter{更新记录}

\verbatiminput{ChangeLog.txt}



	% 以下为正文之后的部分。
	\backmatter

	% 致谢。
	% vim:ts=4:sw=4
%
% Documentation for pkuthss.
%
% Copyright (c) 2008-2009 solvethis
% Copyright (c) 2010-2012 Casper Ti. Vector
%
% This work may be distributed and/or modified under the conditions of the
% LaTeX Project Public License, either version 1.3 of this license or (at
% your option) any later version.
% The latest version of this license is in
%   http://www.latex-project.org/lppl.txt
% and version 1.3 or later is part of all distributions of LaTeX version
% 2005/12/01 or later.
%
% This work has the LPPL maintenance status `maintained'.
% The current maintainer of this work is Casper Ti. Vector.
%
% This work consists of the following files:
%   pkuthss.tex
%   chap/copyright.tex
%   chap/abstract.tex
%   chap/introduction.tex
%   chap/chap1.tex
%   chap/chap2.tex
%   chap/chap3.tex
%   chap/conclusion.tex
%   chap/encl1.tex
%   chap/encl2.tex
%   chap/acknowledge.tex

\chapter{致谢}

感谢北大未名 BBS 的 MathTools 版和 Thesis 版诸位同学的支持。
特别感谢 pkuthss 模版的最初创作者 solvethis 网友,
以及不断地对 Casper 提出的诸多问题予以解答的 cauchy 网友 :)


	% 原创性声明和使用授权说明。
	\cleardoublepage
\section*{北京大学学位论文原创性声明和使用授权说明}
\vfill

\section*{原创性声明}

本人郑重声明:
所呈交的学位论文,是本人在导师的指导下,独立进行研究工作所取得的成果。
除文中已经注明引用的内容外,
本论文不含任何其他个人或集体已经发表或撰写过的作品或成果。
对本文的研究做出重要贡献的个人和集体,均已在文中以明确方式标明。
本声明的法律结果由本人承担。
\vspace{2.5em}\par

\rightline
{%
	论文作者签名:\hspace{5em}%
	日期:\hspace{2em}年\hspace{2em}月\hspace{2em}日%
}
\vfill

\section*{学位论文使用授权说明}
\vspace{-1em}\par
\centerline{\zihao{-4}(必须装订在提交学校图书馆的印刷本)}
\vspace{1em}\par

本人完全了解北京大学关于收集、保存、使用学位论文的规定,即:
\begin{itemize}\denselist
	\item 按照学校要求提交学位论文的印刷本和电子版本;
	\item 学校有权保存学位论文的印刷本和电子版,
		并提供目录检索与阅览服务,在校园网上提供服务;
	\item 学校可以采用影印、缩印、数字化或其它复制手段保存论文;
	\item 因某种特殊原因需要延迟发布学位论文电子版,
		授权学校~\Square\nbs{}一年~/\Square\nbs{}两年~/~%
		\Square\nbs{}三年以后在校园网上全文发布。
\end{itemize}
\par(保密论文在解密后遵守此规定)
\vspace{2.5em}\par

\rightline
{%
	论文作者签名:\hspace{5em}导师签名:\hspace{5em}%
	日期:\hspace{2em}年\hspace{2em}月\hspace{2em}日%
}


\end{document}

